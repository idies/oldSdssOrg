\documentstyle[11pt]{article}
\textwidth 7.5truein
\hoffset=-1.25truein
\voffset=-1truein
\textheight 9.3truein
\newcommand{\htmladdnormallink}[2]{
\begin{rawhtml}
<a href=#2>#1</a>
\end{rawhtml}
}
%, \begin{rawhtml} #2 \end{rawhtml}}
\setcounter{tocdepth}{1}
\begin{document}
\title{Scientific Requirements and Scientific Commissioning for the SDSS}
\author{Michael A. Strauss, Princeton University}
\maketitle

\tableofcontents
\parskip 0pt
\section{Introduction}
This document outlines the top-level scientific requirements
of the Sloan Digital Sky Survey, and describes how these requirements
can be tested and monitored during the scientific commissioning and 
operations.  For each requirement, its scientific basis is
given (and whenever appropriate, a reference is given as well).  The
tests that need to be carried out during scientific commissioning to
confirm that these requirements are met are
outlined.  The continuing 
tests that can be done on the data to confirm that these
requirements are met are also described (this is an aspect of quality
assurance).  The emphasis everywhere is on requirements that can be
confirmed with well-defined tests with relatively small quantities of
data; without this emphasis, it will be impossible to carry out the
commissioning in a finite amount of time.  

There already exist a number of documents, mostly on the Web, laying
down requirements for various aspects of the survey.  The present
document is largely based on the requirements that have been laid down
previously.  There is a 
\begin{rawhtml}
<a href=http://www.astro.princeton.edu:81/requirements/README.html>web
page</a>\end{rawhtml},
which 
includes links to these various previous documents (as well as the
present document).  Every attempt has been made not to contradict
previously adopted requirements.

In particular, there already exists a 
\begin{rawhtml}
<a href=
http://www-sdss.fnal.gov:8000/dss/doc/protected/combined.req.html>comprehensive requirements
document</a>\end{rawhtml} for SDSS 
software.

In addition, there exists a number of documents which have been written
about the test year itself; they can be found from the 
\begin{rawhtml}
<a href= 
http://www.astro.princeton.edu:81/test/index.html>test year page</a>\end{rawhtml}.  In particular,
see the latest version of the 
\begin{rawhtml}
<a href=
http://www.astro.princeton.edu:81/test/test_1998.ps>test
year plan</a>\end{rawhtml} (although this
document needs work).  This document
has much overlapping content with the present document, but sorts things
in a somewhat orthogonal way.  

  After summarizing the parameters of the survey, this document
systematically goes through each of the major areas of quantitative
requirements.  For each, it outlines the following:
\begin{enumerate} 
\item The quantitative requirements themselves.  For most of the
sections below, there are several distinct numbered requirements. 
\item The scientific justification for the requirement.
\item The consequences of not meeting the requirement; that is, what
our fallback position should be if we find we are not meeting it, or
what scientific goals we would not be able to reach. 
\item The tests we must do during the scientific commissioning to
convince ourselves that we are meeting the requirement, including both
the data that need to be obtained, and the analyses to be done with
these data.  This is really the incorporation of the test year plans
into the present document.
\item Who has been given the authority to carry out these tests, and
who has been given authority to sign off on them.  Generically, this
authority is given to the Science Commissioning Committee (SCC), which
reports directly to the Project Scientist, the Director for Program
Management, and the Project Director.  We have indicated individuals
throughout who have these responsibilities at the time that this was
written; this perhaps should be replaced with their titles, so that
the document stays valid if individual's jobs change. 
\item The timescale within which these tasks will be done. 
\item The resources required to carry out these tasks. 
\item Quality Assurance; how we can confirm that these requirements
continue to be met during routine operations.  Data that don't do so
will be declared not to be of survey quality. 
\item Enhanced goals, as appropriate.
\item Further test year tasks.  For any given area, it is likely that
there will be test year tasks 
that cannot be explicitly expressed in terms of quantitative
requirements.  This may include pieces of software that can only be
written with real data in hand, or scientific analyses needed by the
pipeline to set relevant parameters.  Then would follow a
breakdown, as above, of what data are needed, what analysis is needed,
who is responsible, what the timescale is, and how we know when we're
done. 
\item What's missing; a to-do list for the requirements document itself.
\item Eventually, this document could be augmented with reports of
progress on each of the tests being carried out, and indications of
when we have confirmed that a given requirement is met. 
\end{enumerate}

Note that not every one of the above items is included for every
requirement. 

In its present form, this document is not complete; I've indicated in
various sections where further work needs to be done.  Many of the
individual sections have been written in collaboration with various
individuals in the SDSS; their names are indicated. 



\section{Science Goals of the Survey}
\label{sec:parameters}
  The basic science goals of the SDSS are outlined in the
\begin{rawhtml}
<a href=
http://astro.princeton.edu:81/sdss-general/msg.611.html>SDSS Principles of Operation (PoO)</a>\end{rawhtml}.  They are as
follows:
\begin{enumerate}
\item  To perform a CCD imaging survey of roughly $\pi$ steradians in
image quality of $\sim 1''$ FWHM for point sources in the
Northern Galactic hemisphere in 5 bands to $u'=22.3, g'= 23.3, r'=23.1,
i'=22.3, z'=20.8$ (5$\sigma$ limit for point sources).  
\item  To perform a CCD imaging survey in the equatorial stripe in the
Southern Galactic hemisphere in 5 bands, roughly 200 square degrees,
to roughly two magnitudes fainter.
\item  To obtain high quality spectra and thus redshifts for a
set of $10^6$ galaxies
and $10^5$ quasars carefully chosen from the imaging data. 
\item  To provide the data in a form to enable state-of-the-art studies of the
large-scale distribution of galaxies and quasars, the
luminosity function of quasars for $0 < z < 5$, and the properties and evolution of
galaxies, and to identify and study galaxy clusters.
\item To finish the survey in a span of roughly 5 years. 
\end{enumerate}

  Although there are myriad other scientific products which the SDSS
will produce, they are either secondary to the above (e.g., Galactic
structure studies) or are necessary intermediate steps for carrying
out the above (e.g., the reddening map); they are described
below, where appropriate. 

More details, and specifications of the hardware, may be found in the
Appendix. 



\section{Image Quality and Throughput}
\label{sec:depth}

{\bf This section was written together with Jim Gunn. }

The depth of the imaging scans is determined by the aperture of
the telescope, the scan rate, the size of the CCD's, the transmission
of the atmosphere, the sky brightness, the reflectivity
of all relevant optics, the quantum efficiency and 
noise properties of the CCD detectors, the 
seeing and image quality, and the efficacy of the
reduction software.  The Black Book estimates the effect of all these
quantities, and finds that the 5
$\sigma$ detection limit for a point source in $1''$ seeing should be
$u' = 22.3, g' = 23.3, r' = 23.1, i' = 22.3$, and $z' = 20.8$.  See
\S~\ref{sec:camera-param} for a brief summary of the design parameters
of the imaging camera.  

\subsection{Quantitative Requirements}

We put requirements only on the parts of the system that we have control
over with the as-built telescope and instrument.  These requirements are
drawn from the requirement documents for the telescope and for the
camera; see the Appendix. 

\begin{enumerate} 
\item {\bf Image Quality on the Imaging Chips}
\begin{enumerate} 
\item {\it The telescope in scanning mode off the equator
should return images with better than $1.1''$ Full-Width-Half-Maximum
(FWHM) in $0.8''$ FWHM free-air seeing.}  The design with perfect optics
and $0.8''$ free-air seeing returns images over the field with FWHMs
between $0.83''$ and $1.04''$, 
with images over $1''$ only at the extreme edges of the field in $u'$
and $z'$.  With optical surface quality corresponding to $0.3''$ RMS
image errors, we should easily be able to meet the $1.1''$ requirement,
and to better it over most of the field.  
\item  {\it The variation in the FWHM of the PSF over the CCD imaging array
should be less than 30\% in 0.8'' free-air seeing, and the variation
over a single chip should be less than 15\%.}  The optical design yields a maximum
variation of FWHM over a given detector of 12\% with 0.8'' free-air
seeing, in the edge $z'$ chip, which also has the worst images.  It
also yields a  variation of the FWHM across the array of 25\%. 


\item {\it During routine operations, the ellipticity of the PSF
should be less than 12\% on all chips.}
The design yields a maximum ellipticity when the array is in perfect
focus of 7\%, though in the edge fields this is very highly
focus-dependent, and rises in some cases to 14\% with the focus off by
only 100 microns (18 microns on the secondary).  {\bf Might this be
affected by differential chromatic refraction in the $u'$ and $g'$
chips at high airmass?} 
\item {\it After correction for the power-law scattering wings, the
energy circle enclosing 90\% of the received energy should have a
radius less than 2.5 arcseconds.} 
When the design images are convolved with a double-Gaussian fit to a
$0.8''$ Kolmogorov seeing profile, the 
circle enclosing 50\% of the energy is in all cases within a few
hundredths of an arcsecond (4\%) of the FWHM. The 90\% encircled energy 
circle is in all cases between 2.0 and 2.2 arcseconds. In the real images,
this diameter is strongly influenced by the power-law scattering wings,
which typically contain between 5\% and 10\% of the energy, roughly an
order of magnitude more than is predicted by diffraction. The
scattering wings can be subtracted in software, and we require that the 90\%
circle be smaller than 2.5 arcseconds once this is done.
\end{enumerate}

\item {\bf CCD read-noise}: {\it The read-noise of the photometric chips
should be below 10 $e^-$ ($u'$) and 20 $e^-$ ($g'r'i'z'$). }
\item {\bf Signal-to-Noise Ratio}:  {\it The system should
reach signal-to-noise 50 for point sources in $1''$ seeing at
magnitudes no brighter than  
18.9, 20.5, 20.2, 19.6, and 18.1 in the five bands, and reach the
detection signal-to-noise limit of 5 at 21.9, 23.2, 22.8, 22.2, and
20.7, both under photometric conditions at airmass 1.4 and with
free-air seeing of 0.8 arcsecond.} These brightness levels are
uniformly 0.2 magnitude brighter than expected theoretically.  Note
that this puts an (indirect) requirement on sky brightness. 
\item {\bf Throughput and Quantum Efficiency}:  {\it A 20.0 magnitude
star scanned at sidereal rate with the photometric chips at an airmass
of 1.4 under photometric conditions should give
more than 1300, 9000, 9000, 6500, and 1500 $e^-$, 
for the five filters, respectively.} These values are 75\% of those
expected theoretically with the typical measured QE of the CCDs.  This
may be thought of as a necessary condition to meet the previous
requirement. 

\item {\it Further requirements COULD be written down on ghosting,
bad columns, scattered light, CTE, full well, and so on.}
One could write down what is expected here, but there is little that
could be done if requirements are not met.  One expects appreciable
ghosts {\it a priori} only in $u'$ and $z'$; one sees appreciable
ghosts only in $u'$. 

\item {\it J. Gunn will write requirements on the performance of
the astrometric chips.}

\item {\bf Seeing:} The worst images over the photometric array for
acceptable imaging data should have a FWHM no worse than $1.2''$. 

\item {\bf Completeness of the Imaging Coverage:} No more than 3\% of
frames, averaged over 1/2 hour of scanning, may be rejected for
non-astronomical reasons (bad seeing, airplane trails, focus problems,
etc.) before the scan in question is marked for reobservation. 
\end{enumerate}

\subsection{Scientific Justification}
Obtaining deep CCD imaging of the surveyed area is one of the basic
scientific goals of the SDSS.  This requires high throughput and good
image quality. This drives the number of objects
detected, the efficiency of star-galaxy separation at all magnitudes,
the signal-to-noise ratio of the images of spectroscopic targets, and
so on.   Good seeing clearly pushes all these requirements in the same
direction. 

The median free-air seeing at the site is believed to be $0.8''$; with
the above requirements, one can image on roughly 1/2 of the dark
photometric nights.  ``Photometric nights'' are defined in the
following section. 

Roughly 3\% of the sky will be masked due to bright stars; the
completeness of the imaging coverage allows an equal amount of sky
missed for non-astronomical reasons. 

\subsection{Consequences of Not Meeting Requirements}
Star-galaxy separation, measures of image shape, and the depth of
the data will all be strongly affected. 

%\subsection{How do we determine if these requirements are met?}

\begin{enumerate} 
\item The image quality is measured for each chip by the on-line Data
Acquisition system. 

\item The image quality is measured for each chip by the Postage Stamp
Pipeline (PSP).  
\item The read-noise on each photometric chip has been measured; all
are known to meet the requirements.
\item The Differential Image Motion Monitor will allow us to measure
the free-air seeing directly, so as to test the image quality
requirements. 
\item We can determine the real signal-to-noise ratio at a given flux
level from multiple observations of a given region of sky. 
\item The throughput of each chip is determined by the Photometric
Telescope operations.  Mamoru Doi's calibration system will also be
available for measurements of the throughput of the camera itself.
\item  Image
quality can be assessed with the PSP once data are taken while
tracking along great circles. 
\end{enumerate}

\subsection{Who's in Charge?}
\begin{enumerate}
\item These requirements are on both the telescope
and imaging camera, so Gunn is responsible for tuning
the hardware to meet these requirements. 
\item The tests to confirm that the requirements are met can be found
in diagnostic plots created by the PSP.  It is the responsibility of the pipeline developers
and production system people at Princeton and Fermilab to examine the
results. 
\end{enumerate}
  
\subsection{Estimated timescale for confirming these requirements}
Fall 1999.  By this point, the integration of the PT
and imaging camera data, the final collimation of the optics, and the
telescope drive system, will all be complete. 

%\subsection{Resources needed for carrying out these tasks}

\subsection{Quality Assurance}
\begin{enumerate}
\item The PSP and astroline/IOP will have diagnostics which can be
used to indicate when the PSF becomes unacceptable. 
\item The PSP will continually measure the throughput of the
camera/telescope, given the calibration from the PT.
\item There must be Quality Assurance tools on the mountain to tell
the observers in close to real time when the PSF becomes
unacceptable.  There must be Quality Assurance tools following the
running of the photometric pipeline that flag data as unacceptable due
to poor PSF, poor signal-to-noise ratio, or poor throughput. 
\item Mamoru Doi's calibration system can be used to measure the
throughput, readnoise, and the gain of all the cameras on the chips. 
\end{enumerate}

%\subsection{Enhanced Goals}

%\subsection{Further Test Year Tasks}

\subsection{What's Missing}

\begin{enumerate} 
\item We need requirements on the accuracy and efficiency of the outputs of the
photometric pipeline (and the errors given for each quantity),
especially quantities used in target selection, such as Petrosian
magnitudes.   See \S~\ref{sec:target} and the software requirements
document for some of this. 

\item We need a requirement on the completeness of the photometric
sample as a function of magnitude (separately for stars and
galaxies). 
%discussion of requirements on completeness of the
%photometric sample.  Steve Kent suggests breaking this down as
%follows, but this needs further work. 
%\label{sec:gaps}
%\begin{enumerate} 
%\item Total completeness (separate requirements for stars and
%galaxies). 
%\item A separate requirement on the false positive rate.
%\item Maximum rate of frames dropped due to non-astronomical
%		causes (bad seeing, problems in astroline, airplane
%trails, dropouts in focus loop, etc).  This requirement can be phrased
%in terms of number of frames dropped per run, or number dropped per
%scanline. 
%\item Maximum consecutive number of frames dropped.
%\item Criteria for declaring a frame not of survey quality.
%
%\end{enumerate}
%\item Further work is needed on defining the observational limits
%beyond which data are unacceptable.  This can be based on some
%signal-to-noise limit based on the observed seeing and sky brightness
%on a frame-by-frame basis, and is driven scientifically by the need
%for star-galaxy separation, and doing accurate photometry in $u'$ for
%quasar target selection. 

\end{enumerate}

\section{Photometric Uniformity and Accuracy}
\label{sec:uniform}
{\bf This section has been written by Steve Kent and Michael Strauss.}

The photometric calibration of the SDSS data is accomplished by
obtaining accurate photometry of stars in numerous ``secondary
patches'' with the Photometric Telescope, which are subsequently
scanned by the 2.5m.  The full calibration process is a multi-step
procedure, and the allowable errors from each step are detailed below.
The error budget here is quite tight, and it is realized that not all
these items will be realizable at the {\em start} of the survey, and
as we learn more, the calibrations may be tightened.  For example,
after several years, one can use the multiple
observations of the primary standards to improve the definition of the
photometric system, and thus improve the photometric accuracy
further.  This is called out explicitly in what follows. 

All errors given are rms unless otherwise indicated. 

\subsection{Quantitative Requirements}

The systematic rms errors in calibrating the 2.5m data to the SDSS
photometric system {\em after two years of routine operation} should be

\begin{itemize}
\item Magnitudes: $< 0.02$ mag rms over the sky in $r'$ band.

\item Colors: $< 0.02$ mag rms in $(g'-r')$, $(r'-i')$, and
$< 0.03$ mag in  $(u'-g'), (i'-z')$, for objects with colors bluer than those of
an M0 dwarf (i.e., $u'-g' = 2.7; g'-r' = 1.2; r'-i' = 0.6; i' - z'
= 0.4$, from Lenz et al.~(1998), ApJS, 119, 121).
\end{itemize}

Magnitudes and colors are computed for any random sample of objects bright enough
that photon shot noise is negligible, over the whole
sky.  At the beginning of the survey, these numbers should be 0.03 mag
in magnitudes and $g'-r'$, $r'-i'$, and 0.04 mag in $u'-g'$, $i'-z'$. 

Note: the values of the errors in magnitudes and colors are the same;
this allows for the fact that often there are correlated
errors in the raw magnitudes which partially cancel when computing colors.

The 2\% (or 0.02 mag) allowed error in the $r'$ magnitude and colors as defined above
has 11 contributions; the error from each contribution is arbitrarily
taken to be $2\%/\sqrt{11} = 0.6\%$, although ultimately it is the
combination of all ten that must be constrained.  For a few of these
contributions, we use a smaller error, where it is justified.  We list the error
budget here; the final two items are end-to-end tests. 

\begin{enumerate}
\item Primary standard star calibration - internal error.

   \begin{enumerate}
   \item Requirement: {\it The uncertainty in the mean magnitude and
$r'$ color for each primary star should be no more than 1\%
	relative to a system defined by any group of ${}> 10$ stars.}
   \item Basis: We use 10 stars to define 3 constants ($a$, $b$, $k$)
	in a typical photometric solution; the system is defined to
	an accuracy $1\%/\sqrt{10-3} = 0.3\%$.  So this item has a
	smaller contribution to the error budget than do the other
	contributions. 
   \item How to test:
	\begin{itemize}
	\item Compute rms residuals for a primary standard relative
	   to the mean from observations over one year, and from this,
	   compute errors in the mean.
	\end{itemize}

   \item Who is in charge: Alan Uomoto
   \end{enumerate}

\item Primary standard star calibration - external error.

   \begin{enumerate}
   \item Requirement: {\it The transformation from the USNO 40'' to
the PT instrumental systems should add no more than 0.6\% random error
to the photometric system, per star.}
%difference in $r'$ magnitude and color
%	for each primary standard as measured by the USNO 40'' and the PT shall not
%	exceed 1\%.}
   \item Basis: We use USNO to set up the standard star system, but will
	use the PT to calibrate the 2.5m.  The photometric systems of
	the two are not identical. 
   \item How to test:
	\begin{itemize}
	\item Obtain parallel calibration of the full set of primary standards
	   with the MT, the PT, and the USNO 40'', and confirm that they are on the
	   same system.  Compute residuals as a function of color, magnitude, and
	   Right Ascension, and look for trends.  
	%The differences in mean for any single star should not exceed
	%1\%.
	%MAS: I took out the above sentence, as it is stated in the
	%requirement itself.
	\end{itemize}

   \item Who is in charge: Alan Uomoto
   \end{enumerate}

\item PT Linearity.
   \begin{enumerate}
   \item Requirement: {\it Uncorrected nonlinearities due to imperfect shutter
	timing and nonlinear CCD/amplifier gain in the PT shall be less than
	0.3\%, measured as the peak error between shortest and longest
	exposure times, and between the faintest and brightest unsaturated stars.}
   \item Basis:  The primary standard stars are observed with short
	integrations at high count rate, while 
	secondary patches are observed with longer integrations at low
	count rate.  Any nonlinearities will give systematic errors in
	the secondary patch calibration. 
	A peak error is more relevant than rms error.  
   \item How to test:
	\begin{enumerate}
	\item Carry out laboratory tests of linearity using a
calibrated light source. 
	\item Carry out laboratory tests of shutter shadowing using a stable
	   light source.  Compare results of shortest and longest
	   exposure times, and measure variation in the throughput
	   across the CCD. 
	\item Observe primary standards at a variety of exposure
	times, and compare results as a function of exposure time. 
	\end{enumerate}
   \item Who is in charge: Alan Uomoto
   \end{enumerate}

\item Nightly PT extinction and zero-point determination.
   \begin{enumerate}
   \item Requirement: {\it The uncertainty of the best-fit
solution to the primary star magnitudes determined over a 1.5 hour
interval should be less than 0.6\% in all 
bands, for airmasses between 1 and 1.75, not including the uncertainty
in the primary star magnitudes themselves.  That is, the scatter of
the standard stars around the best-fit solution, divided by the square
root of the number
of degrees of freedom, should be less than 0.6\%, after correcting for
the known uncertainty of the primary star magnitudes.}
   \item Basis: This test accounts for errors from uncorrected variations
	in extinction during a night and any other uncalibrated
	effects.  Airmass 1.75 is the highest airmass at which survey
	scans will be taken. Note that it does {\em not} include the effects of
	uncertainties in the standard star magnitudes themselves; that
	is covered in the first two items above. 
   \item How to test:
	\begin{enumerate}
	\item Carry out observations of the primary standards only within
	   a small area on the PT CCD, so as to minimize the effects
	   of flatfield variations (see next item below). 
	   Compute the residuals of each star from the global photometric solution
	   on each night.  Compute the mean and standard deviation of
	   the residuals over ten or more nights, and correct for
	   contributions from errors in the standard star calibrations.
	\item Observe twenty or more primary standards with the PT on
	   one night.  Divide observations into two groups randomly, by
	   time, and by azimuth. Apply the photometric solution
	   derived from one group to the other, and compare with the
	   known values. 
	   Repeat for 10 nights. 
	\item Further tests defined by Alan Uomoto. 
	\end{enumerate}

   \item Who is in charge: Alan Uomoto
   \end{enumerate}

\item Flatfielding of PT CCD patches.

   \begin{enumerate}
   \item Requirement:  {\it The rms variation in photometric
calibration of any star measured in different positions of the PT CCD
shall be no worse than 0.6\%.}% (averaged over $3\times 3$ section).
   \item Basis: Standard stars are observed in a small area of the CCD,
	while secondary stars in the patches cover the entire area.  About 1/3
	the area of the PT CCD is used in calibrating any 2.5m scanline.
	The true error in the flatfield over this region 
	%in the section where the standard stars are observed
	is NOT averaged out. {\it It is possible that we can do
	substantially better on this item.}
   \item How to test:
	\begin{enumerate}
	\item Expose a single star in a $9\times9$ grid over the PT CCD.  Compute
	   the rms variation in the calibrated magnitude relative to
	   the mean.
	\item Obtain PT frames of secondary fields spaced by one-half the CCD width.
	Compute the mean difference in the calibration in overlap regions.
	\end{enumerate}
   \item Who is in charge: Alan Uomoto
   \end{enumerate}

\item Calibration of PT patches.
   \begin{enumerate}
   \item Requirement: {\it The zeropoint for photometry of faint PT stars anywhere
	in a secondary patch shall have an
	rms error no worse than 0.6\% relative to the mean zero-point for
	bright stars in the patch.}   This error includes sky subtraction
	%aperture correction, 
	and any
	algorithmic systematic errors, but excludes flatfielding errors and
	photon statistics.  It is presumed that these contributions to
	this error are systematic and
	thus do not cancel out for many stars in a patch. Here
   \item Basis: Sky subtraction and aperture corrections are important
potential sources of error for faint stars.  The dividing line between
bright and faint assumed here is a tunable parameter in MTpipe; we use
this definition consistently throughout this section.  But we might
consider ``bright'' to refer to stars bright enough that photon
statistics in their photometry is negligible, while ``faint'' are the
majority of objects at relatively low signal-to-noise value, which
will be used for the calibration of the secondary patches. 
   \item How to test.
	\begin{enumerate}
	\item For a few patches, measure a field with lots of bright
	   stars repeatedly, moving the field by 1/2 CCD between exposures.
	   (This test overlaps the flatfielding test above).  Compare
	   the average difference in calibrations between the
	   exposures for the bright and faint stars in the field,
	   respectively.  
	\item In overlaps between the PT patches and 2.5m imaging fields,
	   compare the calibration zero points between bright and faint stars.
	   That is, look at the solution for the aperture correction
	   as a function of magnitude. 
	\end{enumerate}
   \item Who is in charge: Alan Uomoto and Doug Tucker
   \end{enumerate}

\item Photometric mismatch between PT and 2.5m CCDs after applying
	a linear color term.

   \begin{enumerate}
   \item Requirement: {\it The rms photometric mismatch between the PT
	and 2.5m imaging camera shall be no worse than 0.6\% for
	main sequence stars of spectral type between O5 to M0.}
   \item Basis: This requirement is just a placeholder to make sure
	that the mismatch is actually quantified.  
   \item How to test: It can be calculated by direct computation,
	 given the known CCD and filter responses, and the SED's of stars.
   \item Who is in charge: Alan Uomoto
   \end{enumerate}

\item Variations in aperture correction within a single 2.5m CCD.
   \begin{enumerate}
   \item Requirement: {\it The photometric pipeline aperture
	correction shall have an internal rms error no bigger than
	0.6\% averaged over column number for arbitrary camera column,
	filter, 
	and seeing between 0.8 and 1.5 arcsec.}
   \item Basis: The aperture correction converts PSF counts to the counts inside a
	large aperture.  Because of PSF variations, this quantity will
	vary across a CCD field.  The photometric pipeline will
	take this variation into account as best it can.  {\it We may
	be able to do substantially better than this requirement.}
   \item How to test:
	\begin{enumerate}
	\item Compute the aperture correction on an individual basis for isolated bright
	   stars.  Compute the rms residuals relative to those
	   actually used by the photometric pipeline, over the full
	   range of columns and for 3 fields spanning a PT patch, and
	   look for trends with column number. 
	\end{enumerate}
   \item Who is in charge: Steve Kent and Robert Lupton
   \end{enumerate}

\item Spatial variation in photometric calibration within a single 2.5m CCD.
   \begin{enumerate}
   \item Requirement: {\it The rms variation in the zeropoint difference between
	PT stars and 2.5m stars evaluated over the full range of
	columns in each 2.5m chip shall be no worse than 0.6\% within
	a single PT patch, for seeing between 0.8 and 1.5 arcsec.}
   \item Basis:  The photometric calibration is affected by scattered light
	in the telescope; this manifests itself as a dependence of the
	zeropoint on column number.  The scattered light pattern may
	vary with telescope and rotator orientation.  {\it We may
	be able to do substantially better than this requirement.}
   \item How to test:
	\begin{enumerate}
	\item Compute the rms difference of the 2.5m and PT magnitude as a
function of column number for bright stars.  Use aperture magnitudes for the 2.5m
measurements, to avoid possible problems with PSF variations across
the chip.
	\item Repeat this exercise for faint stars at a Galactic
latitude which has enough stars to beat the photon noise per star in
the PT measurements.  
	% use PSF magnitudes.  
	%MAS: Steve, I believe we decided to drop the reference to PSF
	%magnitudes, right?
	This allows us to confirm that the bright and faint stars are
	on the same photometric system. 
	\item Repeat the above with the telescope at a range of rotator orientations
		and elevations.
	\end{enumerate}
   \item Who is in charge: Steve Kent
   \end{enumerate}

\item Time variation of the aperture correction along a stripe:
   \begin{enumerate}
   \item Requirement: {\it The photometric pipeline aperture
	correction for each chip shall have an rms error no bigger than
	0.6\% averaged over one hour, for arbitrary camera column,
	filter, 
	and seeing between 0.8 and 1.5 arcsec.} 
   \item Basis: Changes in seeing and focus can cause cause variations
	in the PSF and hence aperture corrections; errors in the
	latter translate directly into photometric errors.  The
	photometric transfer from a calibration patch to an arbitrary
	point in a 2.5m scan includes a term ``apCorr(point) -
	apCorr(patch)''; the rms error of this term is $\sqrt{2}$
	times the rms error at a single point.  
   \item How to test:
	\begin{itemize}
	\item Compute the aperture correction on an individual basis for isolated bright
	   stars.  Compute the rms residuals relative to those
	   actually used by the photometric pipeline, over the full
	   range of columns and for 3 fields spanning a PT patch, and
	   look for trends with row number. 
	\end{itemize}
   \item Who is in charge: Steve Kent
   \end{enumerate}

\item Time variation of photometricity along a stripe:
   \begin{enumerate}
   \item Requirement: {\it The rms variation in the photometric
zero-point shall
	not exceed 0.6\% over an arbitrary time span for any camera column.}
   \item Basis:  This test is an end-to-end test of the transfer of
calibration from the PT to the 2.5m; additional errors may come from
atmospheric extinction variations and other uncalibrated sources.
   \item How to test:
	\begin{itemize}
	\item Observe PT patches overlapping 2.5m scans with
	   1/2 hour separation.  PT patches shall have been measured in close
	   proximity in time and reduced in a single excal run with one extinction
	   coefficient.  Compute the rms of extinction-corrected zero points.
	   Use data with minimal change in seeing to eliminate it as a
	   variable.  Use big aperture magnitudes on the 2.5m data to
	   eliminate the effect of variable PSF. 
	\end{itemize}
   \item Who is in charge: Steve Kent
   \end{enumerate}

These next requirement is not independent of the above; rather,
it combines a number of the requirements just listed.

\item Additional end-end test
   \begin{enumerate}
   \item Requirement: {\it The rms variation in multiple detections of
the same bright object (for which shot noise is negligible) in
independent scans shall not exceed 2.4\%.} 
   \item Basis:  This test is an end-to-end test of the PT to 2.5m
	transfer, and tests the internal consistency of calibrations.
	However, it is
	largely immune to external errors in the primary standard star
calibration, nonlinearities in the PT, and photometric mismatch
between the PT and 2.5m. 
   \item How to test:
	\begin{enumerate}
	\item Obtain a single 2.5m scan perpendicular to the main survey
	   stripe pattern.  Compute rms difference for objects detected in
	   common.  The figure 2.4\% presumes that errors between different
	   scans are statistically independent (which is almost but not
	   quite true).
	\item A less complete test is simply to compare results for
	several observations of a given scan, in the same direction. 
	\end{enumerate}
   \item Who is in charge: Steve Kent
   \end{enumerate}

\end{enumerate}

\subsection{Scientific Justification}
The requirements above are driven by the science goals outlined here. 

\begin{enumerate} 
\item Science of large scale structure from the galaxy and quasar
redshift surveys shall not 
be significantly affected by photometric errors (including the effects
of Galactic extinction).  Errors in
photometric zero point cause errors in galaxy counts that
mimic large scale structure in any apparent magnitude limited sample.
On angular scales comparable to that of the whole survey, an error
budget to the galaxy number counts in the 
spectroscopic survey is roughly as follows:
\begin{enumerate}
\item 2\% True large-scale structure
\item $<4$\% Uncorrected Galactic extinction
\item 1\% Uncorrected errors in completeness corrections
\item 2\% Systematic photometric error.
\end{enumerate}

\item For quasar target selection:
\begin{enumerate} 
\item Photometric errors shall not contribute appreciably to the
number of false negatives. 

Those quasars which have colors that put them close to the
stellar locus will not be observed spectroscopically.
Systematic errors in photometry cause the region in color-color space 
that is omitted to wander relative to the locus of all quasars, making
the selection of quasar targets non-uniform.
It is required that the variation in incompleteness and types of
quasars that are selected be small.  The error is TBD.

\item Photometric errors shall not contribute appreciably to the
number of false positives. 

Systematic errors in photometry cause the measured stellar locus
to wander relative to the fixed selection criteria, increasing the
number of stars falsely identified as quasars.  Similarly, statistical
errors in photometry cause the stellar locus to broaden, scattering stars
into the region of color-color space occupied by quasars. This is
particularly important at $z \approx 2.8$, where the quasar locus
passes particularly close to that of stars. 

The intrinsic rms width of the stellar locus in color space is 0.04
mag at its fattest.  We wish that this increase by no more than 10\%
to keep the quasar false positives to a minimum.  If we presume
Gaussian errors, then the maximum 
tolerable increase in observational error (per color projected
perpendicular to the stellar locus) is $\sqrt{1.1^2-1}\times 0.04 =
0.018$ mag.  This then can be taken as the acceptable error in $g'-r'$
and $r'-i'$.  The $u'-g'$ error can be twice as large.  

Stars redder than M0 have a larger intrinsic cosmic scatter in their
colors than do bluer stars; also, their spectra are complex and cause
the linear relationships used in the color terms in the photometric calibrations 
to break down.  Loosening the requirements on photometric accuracy for
these red objects is
not likely to compromise science.

References:
\begin{enumerate}
\item	Notes by S. Kent presented at Oct 1992 SAC meeting.
\item	Don Schneider, private communication
\item	Heidi Newberg, private communication. 
\begin{rawhtml}
<a href= http://www-sdss.fnal.gov:8000/~heidi/simreport3.html>Report for Simulation
\#3</a>\end{rawhtml}
\end{enumerate}

\end{enumerate}

\end{enumerate}

\subsection{Consequences of Not Meeting Requirements}

Systematic magnitude errors will strongly affect the large-scale
structure core-science goals of the SDSS.  Random magnitude errors, if
they are quantified, can be taken into account in large-scale
structure studies.  Both
systematic and random color errors strongly affect the completeness,
efficiency, and uniformity of the quasar survey, with a deleterious
effect on the quasar science goals. 

\subsection{Who is in Charge?}
Alan Uomoto has overall responsibility to verify that the outputs of
Photometric Telescope are correct.
Steve Kent is in charge of end-to-end quality, due to multiple people involved
in different sets of tests.  See above for specific responsibility for
specific tests.

\subsection{Quality Assurance}
Quality assurance will be conducted in multiple places:

\begin{enumerate}
   \item Online monitoring:  A set of frames in one filter will
	be analyzed on the PT DA computer to determine a quick,
	rough zero point during the night.  This information is
	not used in data processing, but will allow the observers to
determine whether a given night is photometric. 
   \item MT pipeline diagnostics.  The major diagnostics are the
	residual errors in the photometric solutions (which provide
	a measure of internal error in the solution) and comparison
	of calibration constants with expected values.  The first
	can be used to reject a night outright and
	is equivalent to test (4) in \S 4.1 above; the second
	diagnostic can be used as a warning.	
   \item Photo pipeline diagnostics include test (9) from above, which
	monitors problems with 2.5m flatfielding, and the change in
the position and width of the stellar locus with field number, which
tests item (10) above, and can be used as a warning diagnostic.
   \item Final calibration diagnostics include test (9) from above,
	and comparison of calibration constants with expected values.
	The first test can be used to detect flatfielding problems
	and reject a run.
   \item Database diagnostics: The principal diagnostic is a
	test of photometric repeatability for objects detected in overlap
	regions of the different scanlines and stripes.  The test largely
	repeats test (12) above, although it does not fairly
	test (8) and (9).  The rms differences should not exceed 3.6\%.
\end{enumerate}

\subsection{What's Missing}
\begin{itemize} 

\item We need to flesh out the requirements on reddening and QSOs
better.  This will improve as we gain experience with testing the
color selection criteria for QSOs from actual spectroscopy.  The
following is a possible statement of the requirement on reddening:

\item The interstellar reddening should be determined, with a maximum rms
error in colors of 1.5\%.  This yields rms errors in $r'$ extinction,
as determined from stellar colors, of order 3\%. 

\end{itemize}


\section{Astrometric Requirements}
\label{sec:astrom}

Requirements on astrometry fall into two categories: requirements on
absolute astrometry in a given band (we refer here to $r'$), and
requirements on the relative astrometry between bands.  We treat the
two separately here.  This section of the document was prepared by
Jeff Pier, 
based in part on the 
\begin{rawhtml}
<a href=
http://www.astro.princeton.edu:81/sdss-astrom/msg.115.html>minutes</a>\end{rawhtml} of a meeting to discuss astrometric
requirements.   For
background information giving the history of astrometry requirements,
see 
\begin{rawhtml}
<a href=
http://astro.princeton.edu:81/requirements/astrom/sdssAstrometryRqmnts.html>this document</a>\end{rawhtml}.  

Each requirement in this section is expressed as an rms figure per
coordinate in units of milli-arcseconds (hereinafter mas) along a great
circle. 

\subsection{Requirements on Absolute Astrometry}
\subsubsection {Quantitative Requirements}
 The requirements in this subsection apply to the positions determined
in $r'$ for point-source objects brighter than $r' = 19$ with spectral
type between O5 and M0. The determination of
positions for very blue, very red, or emission-line objects may, and
certainly occasionally will, fail to meet the requirements, due to
differential chromatic refraction.

It is expected that the requirements herein will be satisfied for each
``acceptable imaging run.'' Here an acceptable imaging run is a scan of
length no less than 20 minutes (exclusive of ``ramp-up'' frames),
airmass $\le 2.0$,
seeing $\le 1.5$'' (FWHM), with the absolute telescope pointing error
no worse than 3 arcsec rms (2-D on the sky) and relative telescope
tracking error no worse than 50 mas (1-D on the sky) on time scales
of 1-10 minutes.\hfill
\break 
\break
There are three different levels of requirement for absolute astrometry:

\begin{enumerate}
\item The ``Drop-Dead'' requirement -- if the astrometry does not meet
this, the run will be deemed unacceptable since core survey science
would be jeopardized:

{\bf The absolute astrometry on the sky should be no worse than
180 mas.}

\item The ``Science Goals'' requirement -- the astrometry should routinely
meet this requirement to enable non-core science:

{\bf The absolute astrometry on the sky should be no worse than
100 mas.}

\item The ``Enhanced Goal'' requirement -- if all goes well ({\it i.e.},
the telescope tracks well, the atmosphere (seeing, anomalous refraction)
are favorable, the catalog density is at or above 10 stars/deg$^2$) the
astrometry should be limited by the atmosphere: 

{\bf Reasonable effort should be expended to attain absolute astrometry
on the sky of 60 mas.}
({\it N.B.} -- the number that appears here is really no more than an
educated guess about the properties of the atmosphere.)
\end {enumerate}

\subsubsection{Scientific Justification}
\begin{enumerate}
\item The requirement of 180 mas ensures that astrometric errors do not
contribute significantly to loss of spectroscopic fiber throughput (see
the ``Black Book,'' Section 10.1, as well as the
\begin{rawhtml}
<a href=http://www.astro.princeton.edu:81/requirements/fiber_spec.ps>requirements on drilling
accuracy</a>\end{rawhtml}).
\item Currently (and soon to be) available all-sky/wide-area astrometric
catalogs can be used with SDSS data to determine proper motions.  The
100 mas
requirement ensures that the accuracy of the proper motion determinations
will be limited by the other catalogs, not by SDSS.

Additionally, SDSS data will be used to identify optical counterparts of
objects detected in other spectral regions. Again, this requirement
will ensure that SDSS positions are not the limiting factor in making
the identifications.
\item There is no specific scientific justification for the enhanced goal.
Rather it is felt that the survey astrometry should be ``as good as it
can be.''  The astrometric performance can be no better than limits imposed by
\begin{enumerate}
\item the effects of the atmosphere (``seeing'' and ``anomalous refraction'');
\item discontinuous telescope motions due, for example, to glitches in 
tracking or in mirror motion;
\item the accuracy and density of available astrometric reference frames
(catalogs).
\end{enumerate}
In nearly ideal conditions, the astrometric accuracy should approach such
limits which are thought (but by no means proven) to be, when added in
quadrature, on the order of 60 mas.
\end{enumerate}
\subsubsection{Consequences of Not Meeting Requirements}
If the ``Drop-Dead'' requirement is not met it will seriously impact
the core science of the Survey; the signal-to-noise ratio of the
spectra will be degraded. If the ``Science Goals'' requirement is not
met, the SDSS astrometry will dominate the error budget in future proper
motion studies. 
\subsubsection{How do we determine if these requirements are met?}
The ideal, direct way to determine astrometric performance is simply to
compare star positions determined with SDSS with those determined
externally.  Since positions of objects in the $r'$ photometric bandpass
are required, it is necessary to compare positions of stars which
are fainter than 14th magnitude (and indeed down to
19th to fully satisfy the requirement).  The only wide-area astrometric
catalogs available for such comparisons with accuracies reliably better
than our requirement of 180 mas are the catalogs constructed from the
POSS-II plates (catalogs of the northern sky are being generated by the
USNO's PMM machine, STScI's digital sky survey and Minnesota's Automated
Plate Scanner).  These catalogs are expected to be accurate to 
about 150 mas at the epoch at which they were taken.

The equatorial
astrometric zones established by USNO for SDSS can be used to judge
performance to higher accuracy.  These 
zones have typical astrometric accuracies of 70 mas at the faint
end ($r' \sim 17 $).  These fields are limited to the equator and
do not allow for a full exploration of telescope tracking parameter
space, and offer only a limited range for exploring atmospheric
variations across the sky.  They do represent, however, our best
external check on astrometric performance.

For the future, the USNO is conducting a ``red lens survey''
which will reach to about 16th magnitude with an accuracy approaching
50 mas.  The southern red lens survey is underway now and should finish
in the year 2000, after which the telescope/detector will be moved to
the northern hemisphere.  Once operational, it  will require an additional
two years to complete the observations.  While the catalog produced
from this survey may be helpful in eventually producing an astrometric
catalog over the SDSS survey area, it will not be available to help
determine the SDSS performance in the first few years of the survey.

But, for now, the alternative is to use a boot-strap approach:
matching detections on the astrometric chips with catalog positions of
astrometric reference stars, then transferring the solution, using bridge 
stars of intermediate magnitudes, to the $r'$ bandpass.
(This is, in fact, how the astrometry solution is done in the first
place.)  This method is somewhat circular but is really the
only choice.

It is difficult to perform external checks to accuracies better than
150 mas or so since SDSS is in the happy position of being unique as
an astrometrically accurate, wide-area, faint-limit survey.

Repeated scans of the same patch of sky will allow for a very useful
consistency/internal check, but such experiments will not be able to
rule out all systematic effects. The number of secondary overlap stars can
be enlarged by scanning a strip once, then scanning again after shifting
the bore-sight by half a chip-width (instead of the normal $\sim$
chip-width) for consistency checking.  Another possibility is to scan
over a given region in different directions.  For example, we could
carry out a few scans in a direction perpendicular to the main scans,
which should have quite different systematics. 

\subsubsection{Quality Assurance}
The following QA will be generated by the Astrometric Pipeline, for use
by Survey Operations to reject or accept the Pipeline run: 

\begin{enumerate}
\item For each CCD separately, the ensemble mean and rms residuals
in both great circle
longitude and latitude from the least-squares fit are
calculated. Separate values will 
be generated for the entire scan, and for each segment of the scan, where
the length of the segment is set in the test year by the expected number
of standard stars per segment. A warning is issued if the mean or rms for
any segment, or for the scan as a whole, exceeds a fiducial value. 

\item For each CCD separately, the number of standard stars matched and
used in the least-squares fit within each segment is tabulated. A
warning is issued if 
the number of stars for any segment is less than a fiducial value. 

\item A measure of the goodness-of-fit for the least-squares fit is
calculated. Whether 
this is a $\chi^2$ value, a run-up-test, or some other value is yet to
be determined. A warning is issued if this value exceeds a fiducial value.
{\it This is not yet implemented.} 

\item A given standard star will typically be detected 2 to 4 times on
the astrometric chips in a single imaging run. These are matched in the
pipeline, and their positions averaged. Matching is performed between the
leading and trailing astrometric chips in each column, and between adjacent
columns (using the positions averaged over the two astrometric chips in
each column). The ensemble mean and rms residuals (in catalog mean place)
are calculated in both great circle longitude and latitude for each
match-up (6 leading/trailing match-ups, 10 adjacent column match-ups).
Separate values will be generated for the entire scan, and for each segment
of the scan. A warning is issued if the mean or rms exceeds a fiducial value. 

\item The change in the zero-order terms for the CCD and dewar offsets,
rotations, and scale factors, relative to the input values from the
``opCamera" file are tabulated. A warning is issued if any values
differ from their fiducial values by more than an acceptable error. 
\item Information regarding the tracking performance of the telescope can
be generated from the fitted focal plane model (e.g., maximum deviation
of the bore-sight perpendicular to the desired tracking path). This
information would not be used to judge the performance of the pipeline.
However, if any quantity exceeded a fiducial value, a warning could be
issued so that the mountain could be informed of a potential problem. 

\item {\it There is no code in IOP to compare results from the leading
and trailing chips, but it would be useful for these purposes.}
\end{enumerate}

\subsubsection{Who's in Charge}
The U.S. Naval Observatory has responsibility for astrometric 
performance.  The team leader is Jeff Pier.  Responsibility for
telescope mechanical and optical performance lies with the telescope
engineering group, French Leger is the leader.  The effort for work on
telescope tracking and pointing is led by Paul Czarapata. 
Responsibility for atmospheric phenomena is beyond the scope of this document.

\subsubsection{Further Test Year Tasks}
The astrometric test year plans can be found off the astrometric
pipeline's homepage 
\begin{rawhtml}
<a href=
http://www-sdss.fnal.gov:8000/astrom/doc/www/testYearPlan.html>here</a>\end{rawhtml}. 

\noindent There is also a related set of engineering/acceptance tests to be
performed on the telescope 
\begin{rawhtml}
<a href=
http://www.apo.nmsu.edu/Telescopes/SDSS/eng.papers/19961009_TelescopeTsts/19961009.html>here</a>\end{rawhtml}.

\noindent There is an extensive list of requirements/specifications for
telescope and telescope tracking performance 
\begin{rawhtml}
<a href=
http://astro.princeton.edu:81/requirements/astrom/telRequirements.html>here</a>\end{rawhtml}.

Deducing the relative contributions to astrometric error from (1) the
atmosphere, (2) telescope, (3) camera, and (4) astrom pipeline+catalogs is
a non-trivial matter.  Here are a few suggestions:
\begin{enumerate}
\item Run simulated data through the pipeline:  if the simulations are
correct(!!), one should be able to gauge the pipeline performance, since
one knows the input parameters, including complicating factors arising
from hardware and/or the atmosphere put into the simulations.  {\it This
kind of testing has been carried out in the past, and with correct
inputs from the simulation, the pipeline returns errors which
are consistent with the input parameters.}
\item Carry out multiple observations of a given region of sky (both
with the telescope parked on the equator, and scanning in great
circles) and compare the astrometric results.  Any differences cannot
be due to problems in the input catalogs. 
\item On several windless nights with excellent seeing, park the
telescope on the equator and scan: the resulting errors 
should be due to the atmosphere or catalog systematics or pipeline problems.
The difficulty here is ascertaining that {\em nothing is moving} (i.e.\ no
movements of telescope optics, telescope axes, wind baffle, camera, no
focus changes, no LN$_2$ fillings, or anything else one can think of).
%\item Scan the same part of the sky simultaneously with both the 3.5 meter with Spicam, and the
%2.5 meter with the photometric camera, and
%compare astrometric residuals.   If the
%residuals display coherence, infer that there is large spatial coherence
%due to the atmosphere.  If not, suspect telescope tracking, flexure somewhere,
%or software.  (Of course, no other camera has the field available to the
%SDSS imaging camera, but one might learn something from a single chip
%scanning simultaneously on an adjacent telescope.)
%\item Combine the last two schemes:  scan with both telescopes parked on
%the equator.
\end{enumerate}

\subsubsection{What's Missing}
The astrometric test year plans need to be incorporated into this
document, and the specific quantitative goals by which we can say that
the system passes each test need to be spelled out.

It would also be useful to work out a full error budget for
astrometric accuracy, in the style of \S~\ref{sec:uniform}. 


\subsection{Relative Astrometry Between Bands}

\subsubsection {Quantitative Requirements}
The requirements of this subsection are described 
\begin{rawhtml}
<a href=
http://www.astro.princeton.edu:81/sdss-astrom/msg.96.html>here</a>\end{rawhtml}.  As in the
absolute astrometry requirements, there are three levels of 
requirement.  Relative astrometry is measured not on an
object-by-object basis, but is averaged over timescales of a frame or
perhaps somewhat longer (to be determined). 
As with the absolute astrometry goals, these are
understood to apply to objects of ``normal'' colors with $r' < 19$. 
\begin{enumerate}
\item The ``Drop-Dead'' requirement -- if the relative astrometry does not meet
this, the run will be deemed unacceptable since core survey science
would be jeopardized:

{\bf The relative astrometry between adjacent colors should be no
worse than 180 mas on the timescale over which the relative astrometry
is determined.}

\item The ``Science Goals'' requirement -- the relative astrometry should routinely
meet this requirement to enable non-core science:

{\bf The relative astrometry between adjacent colors should be no worse than
100 mas.}

\item The ``Enhanced Goal'' requirement. 

{\bf Reasonable effort should be expended to attain relative astrometry
between adjacent colors of 40 mas.}
\end{enumerate}

\subsubsection{Scientific Justification}
\begin{enumerate} 
\item Relative astrometry of 180 mas is adequate for merging of
objects, and is acceptable for deblending. 
\item Relative astrometry of 100 mas is ideal for deblending, and also
minimizes systematic errors in aperture photometry colors. 
\item Relative astrometry of 40 mas allows Kuiper Belt objects to be
recognized due to their parallax between the images of different
colors of a given frame. 
\end{enumerate}

\subsubsection{Consequences of Not Meeting Requirements}
If the drop-dead goal is not met, the deblender will have to be run
with much looser constraints on distinguishing close blends, with
strong consequences for studies of close pairs of objects, and
rejection of moving objects (asteroids). 

\subsubsection{How do we determine if these requirements are met?}
The photometric pipeline delivers the mean offset between frames (now
done only to Level 0),
and the error thereof.  {\it Question: Are these quantities which will
be stored in the database?}  This quantity is used as quality
assurance as well. 

\subsubsection{Who's in Charge}
As with the absolute astrometric requirements, the U.S. Naval
Observatory has responsibility for this requirement as well.  The
input of the photometric pipeline, in the form of Robert Lupton and
Zeljko Ivezic, is relevant. 

\subsubsection{Further Test Year Tasks}
We need to determine whether a single frame is the
optimal timescale to measure the mean astrometric offset between
frames, given the nature of the 
atmosphere.  If it is longer, the calculation cannot be done within
frames, and it will have to be rethought.  However, it is probably
true that in this case, we can go ahead with the current code within
the astrometric pipeline for determining the relative offsets between
colors; this probably meets the drop-dead requirement already. 



\section{Target Selection}
\label{sec:target}
{\bf This section has been written by David Weinberg and Michael Strauss.}

The ``target selection requirements'' in effect impose requirements
on the data reduction pipelines (photo in particular), the target
selection pipeline, and the target selection algorithms, given the
properties of data returned by the telescope/camera/spectrographs
under survey imaging conditions.  The primary ``adjustables'' are
the algorithms themselves and the definitions of acceptable
observing conditions for survey imaging and spectroscopy.

The primary requirement is that spectroscopic target selection be
uniform enough that uncertainties in the selection function will
not be the limiting factor in studies of galaxy clustering,
quasar clustering, or distribution functions (such as the luminosity
function) of the galaxy and quasar populations.  However, a
requirement worded in this form is essentially impossible to
check prior to completing and analyzing the SDSS.  We therefore
list quantitative, testable requirements below with the hope
that meeting these requirements will also mean meeting this
underlying requirement.  A secondary desideratum for the target
selection algorithms is that they select a broad range of galaxy
and quasar types. In other words, the main galaxy and quasar
samples should be as ``complete'' as possible to a given apparent
magnitude limit, to the extent that this completeness can be
accomplished without compromising the uniformity of the samples
or being too inefficient. 

The sample selection listed below will involve ``fuzzy'' boundaries.
For example, close to the nominal photometric limit of the galaxy or
quasar sample, the probability that an object is chosen is not a
step-function of magnitude, but rather a slowly varying function.
This gives target selection some robustness to errors in the
photometry, and allows exploration of the sample selection near the
sample boundaries. 

The three main science samples are:
\begin{enumerate}
\item A selection of galaxies to a magnitude limit in $r^\prime$, with
an additional cut on surface brightness to ensure high spectroscopic
completeness.  The current plan is to use Petrosian magnitudes and half-light
surface brightness as the selection parameters, with auxiliary
selection with fiber magnitudes.
\item  An auxiliary selection of galaxies determined by photometric
redshifts to be particularly luminous and red, going approximately 1.5 magnitudes
fainter in $r'$ than the main galaxy sample. 
\item  A selection of quasars based on their optical colors, 
apparent magnitudes, and 20 cm radio fluxes (from the FIRST survey).  For the
purposes of this discussion, a
quasar is defined as any extragalactic object whose optical light is
dominated by an unresolved core with at least one of the following
properties: 
\begin{enumerate} 
\item A non-stellar continuum;
\item Strong, broad (FWHM${}>{}$500 km s$^{-1}$) emission lines; or
\item Strong high-ionization emission lines. 
\end{enumerate}
The default plan for quasar target selection divides the northern
survey area into two parts, according 
to the predicted star density from the Bahcall-Soneira models.  In the
central, low-density region (the ``cap''), the aim 
will be the highest possible uniformity and the most complete possible
quasar catalog, to an apparent magnitude limit (probably in the $i'$ band),
with a reasonable efficiency (65\% has been the fiducial).  In the
higher stellar density skirt, we presume that the stellar
contamination will be much higher (although it is possible that this
is not the case; this remains to be seen).  In this case, the goal is
to get the relatively rare, 
but scientifically valuable brighter quasars because they can be
followed up in high-resolution studies, and because they allow us to
explore the high-luminosity end of the luminosity function.  This will be done 
by accepting a lower efficiency and less complete sample, but overall
a smaller density of targets than in the cap. 
High-redshift ($z > 5$) quasars are both very rare and are
more scientifically valuable than lower-redshift quasars, and thus a much lower efficiency will be
acceptable.  
\end{enumerate}

The current descriptions of these algorithms are given in
\begin{rawhtml}
<a href=
 http://astro.princeton.edu:81/documents/galaxy_selection.ps>``Galaxy and Cluster Selection Algorithm for
SDSS''</a>\end{rawhtml}, 
and 
\begin{rawhtml}
<a href=
 http://astro.princeton.edu:81/sdss-qsos/msg.57.html>``Quasar Target
Selection Algorithm''</a>
\end{rawhtml}; 
many further explications and analyses can be found via the
galaxy and quasar working group mailing lists.  Algorithms should not
be constrained by coding convenience; it is fair to say that the
target selection algorithms are as important as those in upstream
pipelines. 


For brevity, we will refer to the objects in the main galaxy sample
as ``galaxies'' and to those in the auxiliary sample as ``BRGs''
(for bright red galaxies).
Note that calibration targets (e.g., spectrophotometric standards)
are not explicitly mentioned here, nor are stellar and
serendipitous science targets; these do not belong in the high-level
requirements document.   Also not described here are requirements on
interactions between targeted objects.  See Steve Kent's 
\begin{rawhtml}
<a href=http://www-sdss.fnal.gov:8000/dss/doc/protected/combined.req.html>software
requirements
document</a>\end{rawhtml}
for these and other details. 

The targets selected for the Southern strip survey will be a superset
of those selected by these algorithms --- e.g., with more generous
magnitude limits, surface brightness cuts, color cuts, and morphology
cuts.

Once targets are selected based on the pipeline outputs in the operational
database, they are assigned to spectroscopic tiles.  Some fraction of the
tiled targets will be lost because they lie within 55" of another target.


\subsection{Quantitative Requirements}
\begin{enumerate}
\centerline{\bf Galaxies:}
\item[1.] {\bf Surface Density of Galaxies:} The mean surface density
of galaxy and BRG targets, averaged over a (not necessarily
contiguous) region of 50 square degrees or larger, will be in the
range $90-110$ per square degree.  
\item[2.] {\bf Redshift Success of Galaxies:} At least 95\% of galaxy
targets (main sample) will yield a measurable, reproducible redshift,
and will indeed be determined to be a galaxy (and not a star at $z =
0$) under standard spectroscopic observing conditions.  The fraction
of galaxy targets which yield a redshift on one spectroscopic exposure under
standard conditions, and not another, must be well-characterized. 
%In repeated spectroscopic measurements of
%the same objects spanning the range of acceptable survey spectroscopy
%conditions, no more than 1\% of the targets will yield a
%redshift one time and not another.
\item[3.] {\bf Completeness of Galaxy Target Selection:} In regions of
the sky that are not masked (due to bright stars, etc.), 
95\% of galaxies whose true magnitude and surface brightness are both at
least 0.2 mag (mag/arcsec$^2$ for surface brightness) above the selection
thresholds will be selected.  
No more than 5\% of galaxies whose true magnitude and surface brightness 
are both at least 0.2 mag below the selection thresholds will be selected.
In between these limits, the dependence of selection probability on
true magnitude and surface brightness will be reasonably
smooth. 
\item[4.] {\bf Reproducibility of Galaxy Target Selection over Range
of Observing Conditions:} 
Consider two photometric scans covering a given
15 square degree region spanning
the range of acceptable survey imaging conditions; let $P_{i,k}$ be
the selection probability assigned to galaxy $i$ in run $k$, and $N_k$
be the total number of galaxies selected in run $k$.  Target 
selection must be robust (although fuzzy), thus $\sum_i|P_{i,1} -
P_{i,2}| \le 0.01(N_1 + N_2)$. 
\item[5.] {\bf Uniformity of Galaxy Target Selection:} Patches of size
15 square degrees at different Galactic and survey latitude and
longitude will yield surface densities of galaxy spectroscopic targets
that are the same to within 50\%.  {\it Surface densities rather than
raw numbers are specified because the fraction of sky that is masked
may depend on stellar density.  The fluctuations given are still a
place-holder; intrinsic fluctuations on this scale due to large-scale
structure are thought to be of order 15\% rms, so the above value may
be a guess of the relevant peak-to-peak variations.}

\centerline{\bf BRGs:}
\item[6.] {\bf Redshift Success of BRGs:} At least 85\% of BRG targets
in the main sample will yield a measured, reproducible redshift under
standard spectroscopic observing conditions.  The fraction
of BRG targets which yield a redshift on one spectroscopic exposure under
standard conditions, and not another, must be well-characterized. 
%In repeated spectroscopic measurements of
%the same objects spanning the range of acceptable survey spectroscopy
%conditions, no more than 5\% of the BRG targets will yield a
%redshift one time and not another.  
\item[7.] {\bf Completeness of BRG Target Selection:} In regions of
the sky that are not masked (due to bright stars, etc.), 
95\% of BRGs whose true (absolute magnitude, rest frame color) are (0.2, 0.1)
above the selection thresholds will be selected.
No more than 5\% of BRGs whose true (absolute magnitude, rest frame color) are
(0.2,0.1) below the selection thresholds will
be selected. In between these limits, the dependence of selection probability
on true magnitude and surface brightness will be reasonably smooth.
\item[8.] {\bf Reproducibility of BRG Target Selection over Range
of Observing Conditions:} Consider two photometric scans covering a 
15 square degree region spanning
the range of acceptable survey imaging conditions.  Let $P_{i,k}$ be
the selection probability assigned to BRG $i$ in run $k$, and $N_k$
be the total number of BRGs selected in run $k$.  Target
selection must be robust (although fuzzy), thus $\sum_i|P_{i,1} - P_{i,2}| \le 0.025(N_1
+ N_2)$. 

\centerline{\bf Quasars:}
\item[9.] {\bf Surface Density of Quasars:} The mean surface density
of quasar {\it candidates} 
in the cap region averaged over a region of 15 square degrees or
larger, 
will be $20$ per square degree. 
The mean surface density of quasar targets in the skirt region will
be $8$ per square degree. {\it The density of high-redshift quasar
candidates also needs to be set.}
\item[10.] {\bf Quasar Target Selection Efficiency:} At least 65\% of
$z < 5$ quasar targets in the cap region will be 
true quasars (i.e., not stars, or non-AGN galaxies) that yield
a measured, reproducible redshift under standard spectroscopic
observing conditions.  The corresponding minimum fraction for
quasar targets in the skirt region will be 40\%.  

%\item[11.] {\bf Redshift Success of Quasars:} In repeated spectroscopic measurements of
%the same objects spanning the range of acceptable survey spectroscopy
%conditions, no more than 4\% of the quasar targets will yield a
%redshift one time and not another. {\it Perhaps this is a bit lax;
%quasar spectra will generally have high signal-to-noise ratio.}
\item[11.] {\bf Quasar Selection Completeness:} 
In a region of sky reasonably well
sampled by existing quasar surveys in the literature, the quasar
target selection should successfully select 90\% of all known quasars
in the region brighter than $i' = 19$. 
\item[12.] {\bf Reproducibility of Quasar Target Selection over Range
of Observing Conditions:} Consider two photometric scans covering a 
15 square degree region spanning
the range of acceptable survey imaging conditions.  Let $P_{i,k}$ be
the selection probability assigned to quasar $i$ in run $k$, and $N_k$
be the total number of quasars selected in run $k$.  Target
selection must be robust (although fuzzy), thus $\sum_i|P_{i,1} - P_{i,2}| \le 0.025(N_1
+ N_2)$. 

\centerline{\bf Miscellaneous:}
\item[13.] {\bf Covering of the Tiling Algorithm:} The tiling
algorithm will fail to tile no more than $10^{-2}$ of 
the selected galaxy and quasar targets.  {\it This was first written down as
$10^{-3}$, which is almost certainly unattainable at reasonable cost.
We have to be especially careful near the edges of a tiled region.}
\item[14.] {\bf Robustness to Photometric Errors:} The target
selection of galaxies, BRG's, and quasars, should be robust to known
photometric errors. 
\end{enumerate}

\subsection{Scientific Justification}

As mentioned in the introduction to this section, the main scientific
justification underlying all of these requirements is that our
measurements of the clustering and distribution functions
of the galaxy and quasar populations should not be limited by
uncertainties in the selection function.  
A second general justification, especially relevant for quasars
and to some extent BRGs, is to identify objects for follow-up studies.
For example, we would like to identify a high fraction of the
bright quasars ($i'<18$) as targets for follow-up high-resolution spectroscopy.
Justifications for the specific quantitative requirements are given
below, labeled according to the requirement numbers above.

\begin{enumerate}
\item[1.] This surface density gives $10^6$ galaxies over $10^4$
square degrees.  The sky coverage and error reflect our
understanding of the fluctuations due to large-scale structure. 
\item[2.] The 95\% threshold ensures that we will get close to our
desired goal of $10^6$ galaxies, with small contribution from stars
selected incorrectly as galaxies.  From the point of view of sample
uniformity, the essential requirement is that the targets that do not
yield redshifts do so for reasons that are independent of the spectroscopic
observing conditions (e.g., they have no useful spectroscopic features within
our bandpass), so that the selection function is simply multiplied
by a constant redshift incompleteness factor.  We clearly need to
characterize the variation of the fraction of objects for which we are
unable to measure redshifts, as this variation will leave an imprint
on large-scale structure measures. 
\item[3.] This is the main requirement ensuring uniformity of the galaxy
sample for large scale structure and galaxy population studies.
In order to determine the selection function, we need to be able to compute
for each sample galaxy the maximum distance to which it would be
included in the sample (or, more generally, the probability of
inclusion as a function of distance).  This is only possible if 
the galaxies that should be included are included and the galaxies that 
should be excluded are excluded.  Note that this wording allows fuzzy
selection boundaries to be implemented.  Note also that this and the
following two requirements ask for consistency in the star-galaxy
separation to the spectroscopic limit. 
\item[4.] This is a secondary requirement aimed at ensuring uniformity
of the galaxy sample, again allowing fuzzy boundaries.  It implies
that the properties of the selected  
galaxies do not depend significantly on the imaging conditions.  This
then puts important requirements on the robustness of the galaxy
photometry, and the star-galaxy separation.  These may well be the
principal determination of the maximum acceptable seeing,
transparency, and sky brightness for the survey. 
\item[5.] This is a third requirement aimed at ensuring uniformity of
the galaxy sample, concentrating on effects that could produce fluctuations
in the efficiency of target selection on angular scales comparable to
that of the survey itself.  Potential effects of this sort include
influence of bright stars on sky subtraction, errors in the a priori
extinction map or the photometric calibration, or hardware effects
that depend on the direction in which the telescope is pointing.
The requirement is fairly weak as worded, due to the substantial
contribution from large-scale structure on these scales. 
\item[6-8.] The requirements and justifications are analogous to those
of [2-4] for the main galaxy sample.  The numbers are looser because
the questions we aim to address with the BRG sample are less detailed,
and because it seems unrealistic to think that we can obtain the same
level of spectroscopic completeness for the BRG sample as for the
main galaxy sample (if we can, we are perhaps being too conservative
in our apparent magnitude limit).   

\item[9.] With the default numbers above, we would get
80,000 quasars in the 5,000 square degree cap if the selection efficiency
in this region is 80\%, and 20,000 quasars in the 5,000 square degree
skirt if the selection efficiency in this region is 50\%; however, the
minimum efficiency numbers listed in [10] are below this.
\item[10.] The combination of this with requirement [9]
determines the number of quasars that will actually be discovered
by the SDSS.  
\item[11.] We would like to have as complete a quasar sample as possible
within the limits of practicality, so that studies of the quasar
population are not limited by selection biases.  We would ideally like
to have had an absolute requirement on the
completeness of quasar target selection, but such a requirement would
necessarily be untestable without a spectroscopic survey of {\em
every} stellar object over a quite large area.  As known quasars have
been selected with a wide range of techniques, this requirement
captures the spirit of the ideal requirement. 
\item[12.] This ensures uniformity of the quasar sample,
especially important for clustering measurements.
\item[13.] The galaxies that are missed because they are not tiled will
have a complicated spatial pattern that depends on details of tiling
algorithm and the observing strategy.  It will be virtually impossible
to correct measures of large scale structure for this effect because
it will be so poorly understood, so we adopt a conservative limit on
the fraction of galaxies that can be missed.  Of course a much larger
fraction of galaxies ($\sim 5\%$) will be missed because of the 
minimum fiber spacing, but this is an effect that is easy to understand
and can be compensated for in a straightforward manner.  Note that we
have not written down an equivalent requirement on tiling {\em
efficiency}; the tiling algorithm gives a one-to-one correspondence
between efficiency and completeness, and the requirement on the
timescale of the survey require that tiling be conducted at high
efficiency. 
\item[14.] This requirement suggests fuzzy boundaries on magnitude
cuts, as described above.   Moreover, as discussed in
\S~\ref{sec:uniform}, the photometric errors are likely to be larger
at the beginning of the survey than in the end; target selection
should take this into account, and allow enough fuzziness that a
complete sample can be defined from the spectroscopic data once
improved photometry is available. 
\end{enumerate}

\subsection{Consequences of Not Meeting Requirements}
As above, these numbers refer to item numbers in the quantitative
requirements subsection. 

\begin{enumerate}
\item[1.] Either the final sample does not contain the desired number
of galaxies or the survey takes longer to complete.  
\item[2.] The final sample contains substantially fewer galaxies than
originally intended, and/or the main galaxy sample is non-uniform in a
way that is difficult to quantify because of its dependence on observing
conditions.
\item[3.] The selection function of the main galaxy sample is uncertain 
because the maximum distance at which a galaxy would make it into the
sample is poorly known.
\item[4.] The selection function of the main galaxy sample is uncertain
because the number and properties of the selected galaxies depend on
observing conditions.
\item[5.] The selection function of the main galaxy sample has coherent
fluctuations that may prevent successful measurement of large scale
structure on scales comparable to the survey.  Failure to meet this
requirement might also indicate a problem that would affect other
aspects of survey science, including those based on BRGs and quasars.
\item[6-8.] Analogous to [3-5], for the BRG sample.
\item[9.] Either the final sample does not contain the desired number of 
quasars or the survey takes longer to complete.
\item[10.] The final sample does not contain the desired number of quasars
(note interaction with [9]).
\item[11.] The quasar sample is missing scientifically interesting classes
of quasars, reducing its usefulness for studies of the quasar population
and for identifying targets for follow-up studies.
\item[12.] The quasar sample is non-uniform in a way that depends on 
observing conditions, limiting the precision of quasar clustering measurements.
\item[13.] Measurements of large scale galaxy clustering are uncertain
because of the unknown influence of tiling incompleteness.
\item[14.] There may be gradients in the target selection samples as a
function of time, as the photometry gradually improves. 
\end{enumerate}

\subsection{How do we determine if these requirements are met?}

\begin{enumerate}
\item[1,9.] After imaging multiple areas, we tune the selection thresholds
(primarily the magnitude limit) in order to get the desired surface densities.
Note that for quasars this will require imaging at a range of Galactic
positions and appropriate averaging, since the density of candidates
will vary with stellar density even though the density of quasars
should not. {\it We need to determine how many such areas are needed.}
\item[2.] Using spectroscopic observations, including multiple observations
of several fields over the full range (and presumably beyond the full range)
of acceptable survey spectroscopy conditions.  We would also imagine
loosening the galaxy target selection criteria, especially on surface
brightness, to see to what low surface brightness we might be able to
go.  This would allow us to do large-scale structure studies as a
function of surface brightness.  One could carry out a long series of
15 minute spectroscopic exposures of a sample of galaxies extending
several tenths of a magnitude fainter than the nominal limit, and ask how many needed
to be co-added to meet the redshift completeness requirement at each
magnitude. 
\item[3.] This the most important of the requirements, at least from the point
of view of galaxy large scale structure studies, and it is also the toughest
to check because in general we do not know the ``true'' values of galaxy
photometric parameters.  There are two useful methods for assessing
whether this requirement is being met:
\begin{enumerate}
\item[(a)] Photometric parameters measured from a 1/4 sidereal, hence deep
imaging scan are taken as ``truth,'' since they are measured at higher
signal-to-noise ratio (although such a test does not check all
systematic effects).   This test requires a deep scan of a significant
area of sky (probably $\sim 10$ square degrees, yielding $\sim 1000$ galaxy
targets, although the exact value remains to be determined) followed
by one or more scans of the standard depth.   
Alternatively, we can co-add multiple frames (or, much easier, the
atlas images) taken of a given region of sky 
scanned at sidereal rate; this requires that the code to co-add atlas
images must be developed. 
\item[(b)] Bright galaxies with well measured target selection parameters
are artificially redshifted and inserted at random locations into the
imaging data.  We then see what fraction of them are recovered as targets
after running through the data reduction and target selection pipelines.
This test will be a lot of work.
\end{enumerate}
\item[4.] Using repeated photometric scans of the same regions under
varying observing conditions.  
\item[5.] By applying the target selection algorithm to scans over
a wide range of positions on the sky.
\item[6.] Same as [2].  
\item[7.] The 10 square degree survey mentioned in [3(a)] would yield
only 100 BRG's, which may be inadequate to explore the BRG target
selection.  
 There are 3 areas to test:
\begin{enumerate} 
\item	The surface density limits and scatter propagation may be
	tested by observing fields without any color cut and with
	a relaxed luminosity cut.
\item Photometric measurements (including known problems with Petrosian
	magnitudes e.g, for cD galaxies) may be tested with a 1/4 sidereal
	scan of ~10 square degrees providing higher signal to noise.
\item The measurements of luminosity and rest frame color may be
	tested with spectra and spectrophotometry of a large number of
targets.
\end{enumerate}
We will probably need imaging and spectroscopy of ~0.5\% of the full
survey to get adequate statistics for the BRGs (500 BRG plus a few hundred
outside the strict BRG limits).
\item[8.] Same as [4].
\item[10.] By spectroscopic follow-up of quasar targets in fields covering
a range of sky positions (since the efficiency may vary strongly
with stellar density). {\it We need to quantify how many fields are needed.}
\item[11.] Scan areas where there have been extensive previous quasar
surveys to $i' = 19$, preferably using multiple methods. 
\item[12.] Same as [2] and [6].
\item[13.] The fraction of targets untiled should be a tunable parameter
in the tiling algorithm itself.  We just have to ensure that this
requirement is met as the survey proceeds.
\item[14.] The fuzziness of target selection cuts are a series of
tunable parameters, which will be set with the worst-case allowable
photometric errors in mind. 
\end{enumerate}

\subsection{Estimated timescale for confirming these requirements}

Commissioning of target selection is the single largest task of the
scientific commissioning period; 
successful completion of these tasks will largely define
the end of the commissioning period and the beginning of the 
survey proper.  Several of the tests require observations that are
at least 1\% and preferably several percent of the full survey
observations, although we will push to make this shorter if at all
possible.   
\begin{rawhtml}
<a href=http://www.astro.princeton.edu:81/sdss-qsos/msg.58.html>Plans for the imaging and spectroscopy needed for quasar
target selection</a>\end{rawhtml} describe
what is needed for quasars. 

\subsection{Resources needed for carrying out these tasks}

These tests will require reduction of large amounts of imaging
and spectroscopic data, using the production system at Fermilab,
and several person-years of analysis and examination of the results.

\subsection{Quality Assurance}

During the commissioning period, we will be able to measure the
standard deviation of the number of galaxy, BRG, and quasar
targets of different types as a function of area, e.g., in
$2.5^\circ \times 5^\circ$, $5^\circ \times 5^\circ$, and
$10^\circ \times 10^\circ$ patches.
The situation will be trickier for quasar targets than for galaxies/BRGs
because the surface density will change systematically with sky
position (because of the varying number of false positives (stars)
with position), so we will need to develop a model for the mean and 
standard deviation of target numbers as a function of position.

We will also be able to measure the distribution function of the
fractions of galaxy/BRG, and quasar targets that are spectroscopically
confirmed as the expected kind of object and yield measured redshifts,
and we will be able to measure the rms difference between photometric
and spectroscopic redshifts of BRGs.

Quality assurance tests for target selection that should be
continued throughout the survey are:

\begin{enumerate}
\item After target selection but before spectroscopy:
\begin{enumerate}
\item Investigate cases in which the number of targets in a given class
is more than $3\sigma$ away from the expected number on any of these
area scales.
\item Monitor the fraction of targets that are successfully assigned a fiber
(not lost because of minimum spacing or some other conflict or error).
It is not clear what our advance expectations should be here, so we
will have to build up experience over time and search for any
large deviations.
\item Examine the number of targets selected per frame as a function
of seeing, sky brightness, reddening, airmass, and so on. 
\end{enumerate}
\item After spectroscopy:
\begin{enumerate}
\item Investigate cases where the fraction of targets that are confirmed
as members of the expected class and yield successful redshifts falls
outside of the expected range.
\item Compute the rms error of photometric redshifts.  Investigate
cases where this error is substantially different from the expected error.
\end{enumerate}
\item As the survey proceeds:
\begin{enumerate}
\item Investigate correlations between the number of selected targets and
the imaging observing conditions (seeing, moon, etc.) or sky position.
\item Investigate correlations between the fraction of successful
redshifts and spectroscopic observing conditions.
\end{enumerate}
\end{enumerate}

\subsection{Who's in charge}
The galaxy and quasar working groups (and in particular, their
respective chairs, Michael Strauss and Don Schneider) are in charge
of seeing that their science targets are properly selected.

\subsection{Enhanced Goals}
  The quasar efficiency will be 80\% for
$i'<19$ quasars in the cap, and 60\% for $i<18$ in the skirt.

\subsection{Further Test Year Tasks}
\begin{enumerate} 
\item The photometric redshift relation must be determined.  This
requires SDSS photometry in regions of sky with large numbers of faint
galaxies with redshifts in the literature.
\item The TBDs on efficiency for quasar target selection must be
finalized; this will then allow us to turn the numbers of quasar
targets into a photometric limit.  More could be written on the quasar
science goals, to make clearer how to design the quasar target
selection algorithm. 
\item The Petrosian parameters, and other parameters associated with
galaxy target selection (cf., 
\begin{rawhtml}
<a href=
http://www.astro.princeton.edu:81/documents/galaxy_selection.ps>the galaxy target
selection document</a>\end{rawhtml}) must
be fine-tuned, and confirmed to be robust to blending, sky level, and
so on. 
\item We must determine a reasonable surface brightness limit for the
main galaxy sample.  We plan a surface brightness cut, in order to get
reasonable signal-to-noise ratio spectra in our 45-minute exposures.
However, low surface brightness galaxies often have strong emission
lines, making it easier to measure redshifts.  
\item Sanity checks should be carried out of the galaxies selected as
Bright Red Galaxies:
\begin{enumerate} 
\item Do they tend to be BCGs?  That is, do they tend to lie in the
centers of rich clusters?
\item Can we actually obtain redshifts of these objects at the
photometric limit?
\end{enumerate}

\item The ROSAT target selection algorithm must be finalized.  It is
not even described in the document above. 

\item It has been suggested that we obtain multiple spectroscopic
observations of a given field, and determine the fraction of galaxies,
BRGs, and quasars that have a redshift successfully measured one time
and not the other. 

\end{enumerate}

\subsection{What's Missing; What needs to be done}
\begin{enumerate}
\item Requirements on the ROSAT target selection algorithm.
\item Incorporation of quasar and galaxy test year plans. 
\item Further refinement of the quasar target selection goals. 
\item Discussion of requirements on FIRST. 
\item Clarification of responsibilities throughout. 
\item Discussion of requirements on targets other than galaxies and
quasars. 
\end{enumerate}


\section{Spectroscopic Requirements}
{\bf This section has substantial input from Don Schneider.} 

See \S~\ref{sec:spectro-params} for a brief summary of the spectrograph
design parameters. These requirements are drawn
largely from 
\begin{rawhtml}
<a href=
http://www.astro.princeton.edu:81/requirements/spectrograph_specifications.ps>this document on spectrograph
specifications</a>\end{rawhtml};
See that document for more details. 

\label{sec:spectro}
\subsection{Spectroscopic Throughput}

\subsubsection{Quantitative Requirements}
\begin{enumerate} 
\item The overall throughput of the spectrographs at all wavelengths
should be 90\% or higher than that given in Figure 11.8 of the Black
Book, for more than 95\% of the fibers.  {\it This includes the
throughput of all components: fibers, the spectrograph optics, and the
CCDs.}

\item The average throughput of the fibers in each 20-fiber harness
 should exceed 90\%, with a minimum in any fiber of 87\% (not
including broken fibers). {\it
This requirement can be found 
\begin{rawhtml}
<a href=
http://www.apo.nmsu.edu/Telescopes/SDSS/eng.papers/19980203_FiberFeed/19980203.html>here</a>\end{rawhtml}}. 

\item The rms fiber-to-fiber throughput variation at a given wavelength
shall not exceed 4\%. 

\end{enumerate}

\subsubsection{Scientific Justification}
 The primary goal of the spectroscopic observations is to determine the
redshift of the observed objects; this demands relatively high signal-to-noise
ratio spectra over a wide wavelength range.  The absolute throughput
requirements are needed to produced the required signal-to-noise ratio
for the most challenging objects (galaxies with brightnesses at the
spectroscopic limit).  If the fiber-to-fiber variations are too large,
the spectra of a considerable fraction of the targets will be of
insufficient quality to determine a redshift.

\subsubsection{Consequences of Not Meeting Requirements}
The throughput of the spectrograph has a direct relation to the amount
of integration time required to obtain a spectrum of sufficient signal-to-noise
ratio.  A lower throughput translates into longer spectroscopic exposures,
thus either lengthening the survey or reducing the fraction of the survey
that can be observed spectroscopically.

\subsubsection{How do we determine if these requirements are met?}

The fiber throughputs can be measured in the lab; the fibers have been
demonstrated to meet specifications cleanly (see 
\begin{rawhtml}
<a href=
http://www.apo.nmsu.edu/Telescopes/SDSS/eng.papers/19980203_FiberFeed/19980203.html>here</a>\end{rawhtml}
for details).   The total system
throughput can be measured from every exposure (in photometric
conditions) from the spectrophotometric standard(s).

\subsubsection{Quality Assurance}
The fiber-to-fiber variations can be monitored in every exposure for
all wavelengths using the flat field.   Throughput curves (relative if not necessarily
absolute) can be obtained on every exposure from the spectra of the
spectrophotometric standards.  It should also be possible to make crude
throughput calculations for every object using the SDSS imaging data.

\subsubsection{Who's in charge}
The Spectroscopic Scientist.

\subsection{Spectroscopic Wavelength Coverage, Resolution and Stability}

\subsubsection{Quantitative Requirements}
\begin{enumerate} 
\item The spectrographs should have continuous coverage from 3900-9100\AA.  
\item The rms deviation of measured wavelength scale from the arc lines shall be
less than 0.1\AA\ (blue spectrograph) and 0.2 \AA\ (red spectrograph) for all
fibers over the full spectral range.
\item FWHMs of unblended arc lines, in pixels, in a given {\em fiber}
will have an rms dispersion of less than 5\% of the mean FWHM.
\item FWHMs of unblended arc lines, in pixels, at a given {\em
wavelength} will have an rms dispersion of less than 5\% of the mean
FWHM at that wavelength.  
\item The FWHM of unblended sky lines in all the spectra will be less than 1.05 times
that of arc lines in the same part of the detector. 
\item The minimum spectral resolution ($\lambda$/FWHM) in a 15-minute
exposure at any wavelength in any fiber is 1800.
{\it These last two items are requirements on instrument flexure (the Black
Book quotes flexure of ${} \le 0.3$ pixel for exposures from the zenith to
airmass of 1.7), on the grating, and on the optics (telescope +
spectroscopic corrector + collimator).}
\item Flat-fielding must be done so as to be insensitive to any
flexure of the spectrographs. 
\item The cross-talk between adjacent fibers (i.e., the fraction of
the light from a given fiber that falls within the aperture of the
adjacent fiber) will be less than
1\%. {\it It remains to be seen what the spectrographs will actually deliver.}
\end{enumerate}

\subsubsection{Scientific Justification}
\begin{enumerate} 
\item[1.] The large spectral range is required to obtain accurate redshifts of
the targets.  Galaxy absorption lines used for redshifts
   lie primarily in the 3900-6000 \AA\ range, which will be found on
the blue chip for low redshift, moving to the red chip for galaxies at
redshifts of several tenths.  Moreover, high redshift quasars
   might be detected only in the 6000-9100\AA\ range.

\item[2.] The accuracy of the wavelength scale is needed to produce galaxy
redshifts of sufficient quality to investigate the distribution and
motions of galaxies on scales of $\approx 1\,h^{-1}$ Mpc or smaller. 

\item[3-5.] The stability requirements place flexure limits on the
spectrograph, preserving the integrity of the wavelength scale.  The
optical design of the spectrograph (see Table 11.2 of the Black Book,
page 11.14) delivers 4\% variations in FWHM diameters of unresolved
lines, after convolution with the 3-pixel aperture of the fibers. 
{\it Perhaps we also need a requirement on encircled energy.}

\item[6.] The minimum resolution matches the typical width of a galaxy
   absorption line. 

\item[7.] If there is appreciable flexure, the object exposures and
flat-fields taken through the fibers will not line up, giving biased
results.  Thus this can be read either as a requirement on the
flexure (also an issue for the previous item), or if flexure turns out
to be large, a requirement that uniformly illuminated flats be taken. 

\item[8.] Cross-talk between fibers gives contaminated spectra, yielding
unreliable redshifts. 

\end{enumerate}

\subsubsection{Consequences of not meeting requirements}

The accuracy of the wavelength scale translates directly into the
accuracy of the redshift measurements.  For any reasonable expected
error, this will be negligible for quasar redshifts, but will
significantly impact both galaxy and quasar absorption line redshifts. Flexure in
the wavelength direction produces a degradation in the wavelength scale
and resolution; flexure in the perpendicular direction reduces the
signal-to-noise ratio of the spectrum and hence the accuracy of the
redshift. Crosstalk degrades the signal-to-noise ratio of the spectra,
and is likely to give biased redshifts. 

Failing to meet the resolution requirement will result in a
significant reduction of the ability to detect and measure absorption
and emission features (especially when the absorption lines become
crowded, as in the Lyman $\alpha$ forest), and to measure the redshift
and velocity dispersion of galaxies and the redshifts of quasar
absorption lines.

\subsubsection{How Do We Determine if These Requirements are Met?}
\label{sec:resolution-tests}

All of these requirements can be determined from the accuracy of the
wavelength solutions from arc lines, and the width of arc and sky
lines measured.  Arcs taken before and after each exposure give a
measure of the effects of flexure. 

The cross-talk can be determined by examining the extracted spectrum
of sky fibers adjacent in the slit to bright stars.

\subsubsection{Quality Assurance}

A monitoring program using the information described in
\S~\ref{sec:resolution-tests} must be put in place.  The properties
of the arc/sky lines that are produced in every series of observations
must be recorded and checked on a periodic basis.

\subsubsection{Who is in Charge?}
The Spectroscopic Scientist. 

\subsection{Spectroscopic Reliability}
{\it This section is waiting for further input from Steve Kent.}
\subsubsection{Quantitative Requirements}
\begin{enumerate} 
\item The coaddition of three 900 sec exposures in photometric
conditions of an elliptical galaxy at redshifts between 0.0 and 0.3
whose $r'$ fiber magnitude is less than or equal to 19.5 will yield a
redshift with an rms statistical error of 30 km/s. 
\item {\it We need a separate requirement here on systematic errors in
redshifts.  This is much more difficult to design a test for.}
\item Correctly classify 95\% (enhanced goal of 98\%) of the
quasars (for the present purposes, defined as objects with at least one
emission line with an equivalent width${}> 10$\AA) with $i' \le 19.5$. 
Of the quasars not identified as
such, 99\% will be classified as ``unknown class" (i.e., not identified
as a galaxy, star, etc).

\item A maximum of 1\% of the Galactic stars will be assigned redshifts larger
   than~0.01 and have redshifts that differ by more than~$1\sigma$ from~0.0.
   A maximum of 1\% of the galaxies with $r' < 18.2$ (Petrosian)
without strong broad emission lines will be classified as quasars.  {\it
This requirement needs work, as it is rather difficult to measure
failure rates that are this small.}
\item The redshifts of BAL QSOs will be determined to an rms accuracy of~0.02 and the
   redshifts of non-BAL QSOs to an rms accuracy of~0.005. 
\end{enumerate}
These quasar requirements are drawn from 
\begin{rawhtml}
<a href= http://www.astro.princeton.edu:81/sdss-qsos/msg.112.html>sdss QSOs
\#112</a>\end{rawhtml}; see
that 
message for enhanced requirements.  {\it These requirements require large
numbers of test spectra with accurate a priori knowledge of the right
answer; we should think of ways to rephrase things to avoid this.}

\subsubsection{Scientific Justification}

Investigations of small scale/cluster structure with galaxies near the
photometric limit require the stated galaxy redshift accuracy.  To efficiently
analyze the data one needs to be certain that the reliability of the
classifications and the redshifts is high. 

\subsubsection{Consequences of not meeting requirements}

Failure to reach the redshift accuracies for the galaxies will
severely limit the investigation of small scale/cluster structure.
The quasar redshift accuracy limit is driven more by what should be
possible than by scientific goals; the primary goals of the survey can
be met with errors several times larger than the stated limit.  Larger
quasar redshift errors would adversely impact areas such as quasar
absorption lines and gravitational lenses.

Unless the classification reliability meets the requirements, it will
not be possible to trust the automated classifications and each
spectrum will have to be examined, and perhaps analyzed, by hand. 

\subsubsection{How do we determine if these requirements are met?}
Simulated spectra will provide an important test of the redshift
software, but this will not be entirely satisfactory; all we can say
is that if the software cannot meet the requirements on simulated
spectra we can be assured that we won't be able to determine redshifts
in the real data.

Observational tests of the reliability can be done by observing the
same (or more than one) field several times and examining the distribution
of measured redshifts for each of the objects.

In addition, we could do multiple short exposures of a given field,
and ask how many need to be co-added before the redshift reliability
meets specifications.  This needs to be fleshed out. 

\subsubsection{Quality Assurance}
Some small fraction of the objects in the overlapping areas should
be observed twice.  Some fraction of the spectra and results will be inspected
by hand on a continuing basis to monitor the performance of the software.

\subsubsection{Further Tasks}
Some of these items overlap with operations tasks, below. 
\begin{enumerate} 
\item We need to determine the minimum number of sky fibers necessary
for adequate sky subtraction.  
\item We need to determine the number of fibers needed for
calibrations (spectrophotometry, reddening, and telluric absorption
correction). 
\item We need to finalize the final set of calibration exposures
needed with each set of spectroscopic observations (dithered and
undithered flat fields, arc, and four-pointing observations for
spectroscopy). 
\item We need to finalize the spectroscopic templates (via PCA,
standard stars, or otherwise) to be used in the cross-correlation. 
\end{enumerate}
\subsubsection{Who is in Charge?}
The Spectroscopic Commissioner. 

\subsection{Spectrophotometric Accuracy}
This section is abstracted from 
\begin{rawhtml}
<a href=
http://www.astro.princeton.edu:81/requirements/spectrophoto.ps>a document by Bob Nichol</a>\end{rawhtml}. 
It has been worded as a requirement on spectroscopic operations.  It
is understood that formal requirements on spectrophotometric accuracy
on {\em each} spectrum are probably unreachable. 

\subsubsection{Quantitative Requirements}
\label{sec:require}

The spectroscopy should be carried out in such a way to allow
spectrophotometric calibrations to be done.  This implies that:
\begin{itemize} 
\item At least three bright ($\sim15$th magnitude) calibration stars must be
spectroscopically observed per SDSS spectroscopic plate. The star
should possess a smooth spectral energy distribution {\it i.e.} F
subdwarf star.
\item The position of each fiber relative to the center of its
target must be determined with an accuracy of $< 0.5''$ for greater
than 85\% of the fibers. 
\end{itemize}


\subsubsection{Scientific Justification}

The above operational requirements are put in with the scientific aim
of yielding an rms error in the relative spectrophotometric calibration
averaged in 200\AA\ chunks over the entire spectrum no more than
10\%. 

The main science driver for SDSS spectrophotometry is the detailed
study of the stellar content of galaxies and its evolution with
environment and redshift.  We also would like to have one of the main
scientific products of the SDSS, the spectra of over one million
objects, be fully calibrated. 

Spectrophotometry for SDSS galaxies would also help in the redshift
determinations of galaxies as it would allow direct fitting, in
wavelength space, of the galaxy data with well-known
spectrophotometric templates.

The requirement on the brightness of the calibration stars is to
ensure that the PT can obtain a high signal-to-noise detection of
the star, through the intermediate-band spectrophotometric filters,
in a short integration time.  The star cannot be too bright, to
minimize the cross-talk between fibers.  Three stars are needed in
case some turn out to be variable, and may be useful in
non-photometric conditions if the atmospheric transparency varies
appreciably across the field of the telescope.  Finally, note that one
can only do a poor job of synthesizing the spectrophotometry from the
combination of the SDSS broad-band photometry, as one has only three
points ($g', r'$, and $i'$) included in the spectral coverage. 

\subsubsection{Consequences of Not Meeting Requirements}

\begin{enumerate}

\item From studies of synthesized galaxy spectral energy distributions
(Bruzual \& Charlot 1996), the rms error in the relative
spectrophotometric accuracy must be less than $10\%$ to conclusively
differentiate between galaxies of different ages and
metallicities. 

\item Spectrophotometric accuracy will greatly degrade as the fiber
position relative to the galaxy becomes more uncertain. 

\item If any plate does not have calibration stars, neither absolute
nor relative spectrophotometry can be done for any of the galaxies on
the plate, except crudely via their broad-band photometry. 

\item If spectrophotometry is used in the redshift determination, and
if spectrophotometry can't be done for all exposures, then we run the
risk of non-uniform completeness in the redshift determinations.

\end{enumerate}

\subsubsection{How do we determine if these requirements are met?}

The relative spectrophotometric accuracy can be determined with
observations of known F subdwarfs with well--determined
spectrophotometry through a series of SDSS spectroscopic fibers in
turn.


\subsubsection{Test Year Tasks}

\begin{enumerate}

\item Observe a star with published spectrophotometry with the SDSS
spectrographs.  Observe the same star with the SDSS camera and the PT
using the spectrophotometric filters. Repeat the experiment in photometric
and non-photometric conditions, and quantify the reproducibility of
the spectrophotometric results, and the added improvement that the
diamond--pattern raster scan gives.

\item Determine the correct exposure time for the
diamond--pattern raster scan; it is now nominally 1 minute per
pointing.

\item Test the target selection criteria of F subdwarfs from the main
SDSS photometric data.  Determine the magnitude range over which they
should be selected. 

\item Determine the required number of spectrophotometric calibration
stars per spectroscopic plate, given variations of atmospheric
transparency across the plate. 

\item Obtain roughly 10 photometric nights of observations of
spectrophotometric standards with the
Photometric Telescope, using the 10 intermediate--band
spectrophotometric filters. This is required to calibrate
the network of spectrophotometric calibration stars.  Note that this
need not be done before the start of operations. 

\item Carry out checks of the accuracy of the red-blue merging of the
spectra in the spectroscopic pipeline, and the 
fiber--to--fiber throughput corrections.
\end{enumerate}

\subsubsection{Who's in charge}

The Spectroscopic Scientist is in charge of overseeing the
spectrophotometric commissioning discussed here, and, in collaboration
with the Project Scientist, determine whether the tests have been
successful or not.

\section{Operations and Strategy}
\label{sec:operations}
{\bf This section was written by Rich Kron.}

See the document 
\begin{rawhtml}
<a href=
http://www.astro.princeton.edu:81/requirements/operation_requirements.txt>here</a>\end{rawhtml}
for an earlier draft of the present. 

\subsection{Duration of Survey}
\label{sec:length}
\subsubsection{Quantitative Requirements}

The survey shall be completeable within 5 years.  {\it This requirement assumes 
that the data meet quality standards described elsewhere.  The spirit of the 
requirement relates to factors that are within our control; if the 
atmospheric conditions are less favorable than we have assumed, the survey 
will take longer, of course.}

\subsubsection{Scientific Justification}

The survey has been designed around the goal of completion within five
years. The cost of the survey depends on its duration.  

\subsubsection{Consequences of not Meeting Requirement}

Skilled people move onto other projects.  We lose credibility with 
funding agencies.  We will run out of money.% (see \S~\ref{sec:money} below). 

\subsubsection{How do we Determine if this Requirement is Being Met? }

The progress of the survey is reckoned by the area of sky scanned and 
accepted into the science database, and the number of spectra obtained (or 
tiles exposed) and accepted into the science database.  This rate is compared 
to the necessary rate, which is derived from the strategy planning tool. 

\subsubsection{Estimated Timescale for Confirming this Requirement }

Tuning the efficiency of operations will take some time - at least a year 
into the survey.  Thus, while we will have rate data immediately, the 
asymptotic rate of covering the sky will not be certain for at least a year, 
and more realistically well into the second observing season.  An
additional complication is the fact that the rate of taking data at
the {\em end} of the survey may be quite a bit slower, as we struggle
to fill up holes in the sky coverage. 

\subsubsection{Resources Needed}

There will be instances where a trade-off between survey speed and 
available SDSS resources will be required (for example, it costs more to 
repair things faster).  The overarching idea is that Operations should have 
the planning tools needed to maximize the amount of ``survey quality" data, 
given the available resources of time, money, and people. 

An important resource is the strategy planning tool; it needs to be
exercised further to optimize long-term efficiency of the survey. 

\subsubsection{Who is in Charge?}

Director of Program Operations. 


\subsection{Safety}
\subsubsection{Quantitative Requirements}

All relevant systems must be critically reviewed to evaluate the potential 
for hazards to humans and equipment.  A comprehensive review will be undertaken of 
each subsystem before routine operations begin, and subsequent reviews 
will be undertaken as necessary.  Any safety issues must be resolved 
quickly and completely.

\subsubsection{How do we Determine if this Requirement is Met?}

A program of routine safety audits will be established.  Reviews and results 
of actions taken will be documented.  A reporting system will be 
established for logging incidents and the response taken. 

\subsubsection{Resources Needed} 

We need to establish an operations system - and culture - that anticipates 
hazards before incidents happen.  An infrastructure needs to be created for 
planning, tracking, and evaluating the safety audits, and properly reacting 
to the reports. 

\subsubsection{Who is in Charge?}

Program Director. 

\subsection{Spectroscopic Plug Plate Delivery Rate}

\subsubsection{Quantitative Requirements}

After the first year of spectroscopic observations, the rate of progress of 
the survey shall never be limited by the availability of plug plates or 
cartridges.  

\subsubsection{Scientific Justification}

Once a supply of plates has been built up, the spectroscopic
observations are within our control.  The requirement simply states
that the usage of the telescope will not be limited by something
within our control.  The mean rate of plate consumption is 50 per
month.  This requirement states that at all times, there be an
adequate inventory of pluggable plates at APO at all reachable LST's
that spectroscopy is not limited by the availability of plates.  Of
course, as we near the end of the survey, we will run out of sky to be
observed.

\subsubsection{Consequences of not Meeting Requirement}

Observing time is lost or used less effectively than otherwise possible.

\subsubsection{Resources Needed}

The timely fabrication of new plates is a major part of what SDSS 
Operations has to do.  The specifics are covered in subsequent 
requirements. 

\subsubsection{Who is in Charge?}

Program Director.

\subsection{Photometric Reduction}

\subsubsection{Quantitative Requirements}

The achievable elapsed time between obtaining photometric data (tapes out) 
and being able to observe that part of the sky spectroscopically (plates in) is 
required to be normally a full year, with 26 days on an occasional
basis (i.e., of order three times per year).  For two eight-hour
scans, this fast track should be as follows:

\smallskip
\begin{tabular}{lrr}
&fast track\\
\hline
ship tapes to FNAL&1\\
pipelines through target selection&15\\
plate design&1\\
plate fabrication&8\\
ship plates to APO&1\\ \hline
total&26\\
\end{tabular}

These figures are based on the following considerations:  

\begin{enumerate} 
\item Tapes written after each night of observation are shipped to
Fermilab such that they arrive the following day, regardless of the day of
the week. 

\item
The mean rate of acquiring new imaging data each 
month is about 24 hours per month, spread over typically 4 nights.  
The data from a given dark run can be processed within 15 days of 
the end of that dark run.  Notice that this includes running all
relevant pipelines, and stuffing the operational database. 

\item Given a decision to tile a certain region of sky, it shall 
be possible to generate hole positions for up to 50 plates within 
1 day of that decision. 

\end{enumerate}

\subsubsection{Scientific Justification}

This requirement is very much in support of the previous requirement
that adequate spectroscopic plates be available on the mountain at all
times. 

Reducing the data at least as fast as the data are obtained prevents the 
processing from limiting the speed of the survey.  Fast processing allows 
errors to be recognized early, thus preventing observing time from being 
wasted.  The specific times are designed to enable spectroscopic 
observations two dark runs after the imaging data are obtained. 

Timely and convenient access to the data by the collaboration is stipulated 
in the Principles of Operation, and is another major part of SDSS 
Operations.

\subsubsection{Consequences of not Meeting Requirement}

The roughly one-month turn-around is intended to avoid losses in 
efficiency from regions of the sky setting in the West, and also to respond 
to strategically important opportunities (filling in gaps, acquiring data at 
extreme declinations, etc.). 

\subsubsection{Resources Needed}

The Science Database needs to be created and commissioned. 

\subsubsection{Who is in Charge?}

Program Director. 

\subsection{Overhead on Observations}
\label{sec:down}
\subsubsection{Quantitative Requirements}

Intervals of dark time can be classified as: time exposing on the sky; time 
lost due to atmospheric conditions; and everything else.  The ``everything 
else" shall be no greater than 320 hours per year (i.e., roughly the time per 
year needed for taking photometric data).  This contingency is intended to 
cover inefficiencies like: instrument exchange time; calibrations during 
dark time; time elapsed between the determination of clear weather and the 
initiation of taking data; photometric ramp time; inefficiency at the end of 
a night; and scheduled and unscheduled down-time due to hardware and
on-line software 
problems, repairs, and maintenance. 

In more detail, the total overhead associated with spectroscopic exposures, 
i.e., the time not used for on-the-sky integration, shall be less than 22 
minutes per exposure.  The time required to switch between imaging and 
spectroscopy (counting from the end of sky integration to the beginning of 
sky integration) shall be less than 30 minutes. 

\subsubsection{Scientific Justification}

This requirement is intended to ensure that the maximum amount of 
astronomically usable observing time is actually available, in order to 
minimize the time-to-completion for the survey. 

\subsubsection{Consequences of not Meeting Requirement} 

This is really a finer-grained version of the requirement in
\S~\ref{sec:length} (and is in fact derived from it), focusing on things that 
are within our control.

\subsubsection{How do we Determine if this Requirement is Met?} 

The night logs should enable the accounting for time to be easily 
reconstructed.

\subsubsection{Resources Needed} 

If we suffer extended periods of unscheduled down-time, then substantial 
resources may be required to meet this requirement.  If the instrument 
exchange times are unacceptably long, then substantial re-working may be 
necessary. 

\subsubsection{Who is in Charge?}

Program Director. 

\subsection{Spectroscopic Reduction}

\subsubsection{Quantitative Requirements}

The elapsed time between acquisition of spectroscopic data and the 
availability to the SDSS collaboration of reduced spectra (fluxes, redshifts, 
classifications) shall be 15 days, with a goal of 10 days.  {\it
Perhaps an elaboration is needed for the case in which both
photometric and spectroscopic data are competing for reduction
resources at Fermilab.}

\subsubsection{Scientific Justification}

Timely and convenient access to the data by the collaboration is
stipulated in the Principles of Operation, and is another major part
of SDSS Operations.  This particular requirement enables QA by
distributed scientists on a timescale shorter than a lunation.

\subsubsection{How do we Determine if this Requirement is Met?}

The time taken from data acquisition to placing the reduced version into 
the Science Database is easy to determine. Less easy to determine is 
whether the design of the database is adequately responsive to the scientific 
goals of the various SDSS institutions. {\it Perhaps we need a
separate requirement on this.}

\subsubsection{Resources Needed}

The Science Database needs to be created and commissioned. 

\subsubsection{Who is in Charge?}

Program Director. 


\subsection{Requirements on the Photometric Telescope}
Note the overlap of this section with \S~\ref{sec:uniform}. 
\subsubsection{Quantitative Requirements}

The Photometric Telescope is required to measure extinction coefficients for 
each hour during photometric scanning; specifically, it shall be capable of 
calibrating intervals as short as 1.5 hours of 2.5-m scanning data.  It is also 
required to obtain accurate photometry of stars that are faint enough
to be unsaturated  in the 2.5-m telescope scans, spaced approximately
every 15 degrees along  each scan.  The timing of the observations of
these patches must not limit  the schedule for data processing; in
particular, the secondary patches have to keep up with (or even be
ahead of) the 2.5m imaging. 

\subsubsection{Scientific Justification}

The density of calibrations, in angle in the sky and in time, is intended to 
be sufficiently great that the photometric precision is not limited by that 
factor.  The timing requirement comes from the principle that the survey 
should be fundamentally limited by 2.5-m telescope operations, as opposed 
to the throughput of any of the other SDSS systems (otherwise, the rate of 
progress of the survey would be potentially limited by too many other 
bottlenecks). 

\subsubsection{Who is in Charge?}

Program Director. 

\subsection{Requirements on Survey Strategy Tools}

\subsubsection{Quantitative Requirements}

\begin{enumerate} 
\item A planning tool used at Fermilab must be implemented that helps determine 
when a tiling solution should be undertaken, and what the specific 
parameters should be (boundaries of the region, etc.).  Having produced a 
tiling solution, a tool must exist for prioritizing the plates in the sense of 
orders placed for drilling, and their shipment to APO. 

\item A planning tool used by the Observers must be implemented that picks the 
best scans to do on a particular night (in particular, the upcoming night), 
or the best sequence of spectroscopic plates.  The tool must be capable of 
making real-time adjustments to the plan, for example if clouds suddenly 
part in the middle of a night. 
\end{enumerate}

\subsubsection{Scientific Justification}

These tools help minimize the time-to-completion for the survey by 
enabling strategic options to be explored. 

\subsubsection{Resources Needed}

Code needs to be developed that, among other things, is tied into the
Operational Database
(because the assessed quality of existing data needs to be known) and has 
good visualization tools. 

\subsubsection{Who is in Charge?}

Program Director. 


\subsection{Adjustment of Spectroscopic Exposure Time}

\subsubsection{Quantitative Requirements}

Spectroscopic exposures obtained in unfavorable conditions (cirrus,
poor seeing, reddening) need to be lengthened a priori, to yield
signal-to-noise ratios close to those one would get for the same
objects under ideal conditions without reddening.  A system must be
implemented to monitor the integration; to estimate the necessary
extra time required; and to adjust the exposure time accordingly.

\subsubsection{Scientific Justification}

The intent is to obtain uniform signal-to-noise ratio in the spectra for a 
particular monochromatic flux. 

\subsubsection{Resources Needed}
Code needs to be developed to take the known airmass and measured
throughput of the atmosphere from the guide fibers, and feed the
information automatically to the program actually running the
spectrographs. 

\subsubsection{Quality Assurance}
The signal-to-noise ratio of the spectra will be tabulated as a
function of apparent magnitude. 

\subsubsection{Who is in Charge?}

Program Director. 

\subsection{Maintenance of Optical Throughput of Components}

\subsubsection{Quantitative Requirements}

\begin{enumerate} 
\item Operations shall monitor the throughput of the separate 
components along the optical train of both the 2.5-m telescope and 
the Photometric Telescope on a timescale that is short compared to 
detectable changes.  Operations shall mitigate any effects that 
could lower the as-delivered instrumental throughput (or other 
performance). 

\item A practical system must be implemented for cleaning optical 
surfaces on a periodic basis as needed.

\item A practical system must be implemented to handle
mirrors for periodic recoating.  The allowable interval that either 
telescope is down for this purpose is 15 days per year. 
\end{enumerate}

\subsubsection{Scientific Justification}

This is required to keep the uniformity of the data high, and in
particular, to maintain high throughput of the optical system. 

\subsubsection{Resources Needed}

A variety of special-purpose hardware and handling fixtures is needed
for monitoring optical surfaces.    A detailed plan for the monitoring
and cleaning must be developed. 

\subsubsection{Who is in Charge?}

Program Director. 


\subsection{Requirements on Spectroscopic Fibers and their Handling}

\subsubsection{Quantitative Requirements}

\begin{enumerate} 
\item There is {\bf *NO*} requirement to design a plate to be used with a particular 
cartridge (i.e., taking into account its particular suite of broken fibers). 

\item A harness should be replaced when more than 5\% of its fibers
become unusable (e.g., by becoming broken),
based on the scientific need to keep the spectroscopic sampling as complete 
as practically possible.  That is, a harness should be replaced when
missing fibers are comparable in number to the number of objects
missed in target selection because of the $55''$ rule. 

\item Operations must implement a suitable procedure for replacing harnesses.

\item The plug plates need to have indicators to guide the pluggers according to 
groupings of 20 fibers (but there is no further requirement concerning 
which fiber goes into which hole).  Other special holes for special fibers 
(e.g., guide star fibers) must also be indicated on the plate. 

\item The association of fiber number with the hole in which the fiber
was plugged must be  accomplished automatically, with no more than one
pair of misidentified fibers per 10 plates. 

\item No more than one fiber per plate can drop out due
to handling, from the time it is put in the fiber mapper, to the time
it is observed on the telescope. 
\end{enumerate}

\subsubsection{Scientific Justification}
These requirements allow efficient spectroscopic observations, and
accurate association of the spectroscopic and photometric data. 

\subsubsection{Who is in Charge?}

Program Director. 


\subsection{Miscellaneous}

\begin{enumerate} 
\item It shall be possible to change the target-selection parameters twice per year 
for Serendipity objects as determined by the Serendipity Working
Group. 

\item It shall be possible for the Serendipity Working Group to give
hand-picked lists of objects to target selection. 

\item One copy of all of the data tapes sent 
to Fermilab as shall be archived at (or nearby) APO for two years or
more.  Up until the  
end of APO observations, the amount of data lost per year due to the 
corruption of archived data shall be economically equivalent to 
less than 0.5 night per year of repeated observations. 

\item Sufficient spectroscopic analysis must be accomplished on 
the mountain to determine whether a plate exposed during the 
previous night can be unplugged so that the cartridge can be used 
for another plate.  The requirement is that at least 80\% of the working
hours for the plugging staff be within the interval 9 AM -- 5 PM local 
time.

\item Sufficient quality-assurance checks must be undertaken at 
the mountain so that no more than 20\% of all of the data sent to 
Fermilab as ``good'' are rejected on the grounds of not satisfying the
TBD science quality and uniformity criteria. 

\end{enumerate}

\subsection{What's Missing}

There are a number of requirements which are not discussed in this
section: 

\begin{enumerate} 
\item A requirement on the maximum allowable size of any contiguous regions not
covered by the imaging at the end of the survey (see the discussion in
\S~\ref{sec:depth}).
\item Requirements on sky brightness, transparency, and seeing beyond
which imaging would not be done.  {\it These could be different for
North and South observing.}  Criteria for determining these limits
could include:
\begin{enumerate} 
\item Depth of photometry; see \S~\ref{sec:depth}.
\item Accuracy of photometry; see \S~\ref{sec:uniform}.
\item Accuracy of astrometry; see \S~\ref{sec:astrom}.
\item Meeting of target selection requirements; see
\S~\ref{sec:target}. 
\end{enumerate}
These will tie into requirements on photometric depth and uniformity,
and on accuracy of target selection; see those sections for more
detail.  A strawman suggestion for the seeing requirement is no worse
than 1.2'' over the photometric array (not counting the $u'$ chips,
which will be strongly affected by differential chromatic aberration
at large airmass). 

\item  Requirements on sky brightness, transparency, and seeing beyond
which spectroscopy would not be done; see \S~\ref{sec:spectro}.  The principal criterion for
determining these limits will be signal-to-noise ratio of the spectra;
see also \S~\ref{sec:target}. 

\item A check that the signal-to-noise ratio improves as expected as
the Southern stripe data are co-added. {\it If it doesn't for some
reason, our strategy for Southern Stripe observing will change.  We
will want to know this sooner rather than later.}

\item Further requirements on regular maintenance of instruments, with
adequate spares on the mountain, etc.  See \S~\ref{sec:down}.

\item A requirement that we be able to acquire spectroscopic fields
quickly even if the telescope does not meet pointing specifications.
Alan Uomoto has suggested using a 10'' Celestron + CCD strapped to the
telescope for this purpose; see 
\begin{rawhtml}
<a href=
http://www.astro.princeton.edu:81/requirements/spectrograph_specifications.ps>here</a>\end{rawhtml}.
This would keep downtime to a minimum. 

\item A requirement on the acquisition of uniformly illuminated flat
fields for spectroscopy, and for the astrometric chips.  

\item A requirement on the maintenance of adequate focus on the
spectroscopic exposures.  This is part of the throughput requirement
on spectroscopy. 

\item Perhaps most importantly, we don't have specific requirements on
QA.  This is incorporated into this document throughout, of course,
but we could focus on it more directly in this section. 

\end{enumerate}

\appendix
\section{Appendix: Basic parameters of the survey}
This section has been written by Steve Kent.  This section tabulates
useful parameters, but is not a set of requirements per se. 

\begin{enumerate}
\item The Northern Survey boundary is defined by an ellipse as drawn
on a polar equal-area projection of the sky centered on the North Galactic Pole.
The ellipse is centered on $\alpha = 185^\circ$,
$\delta = 32.5^\circ$ (J2000), with axes $130^\circ \times
110^\circ$.  The major
axis is rotated clockwise by 20$^\circ$ relative to an east-west orientation
(and thus, is tilted to higher declination at greater right ascension).
\hfil\break
Basis: The ellipse is aligned to avoid interstellar reddening as much as
possible, and to maximize the length of photometric scans.  The ellipse
center in declination is chosen to center one stripe on the Celestial Equator.

\item There are three stripes that will be done in the Southern
Galactic Cap.  They are roughly given by: 
\begin{enumerate}
\item Stripe 76: $\alpha = 336^\circ$ to $\alpha = 60^\circ$, $\delta =
15^\circ$ at $\alpha = 0^\circ$.
\item Stripe 82: $\alpha = 310.5^\circ$ to $\alpha = 60^\circ$, $\delta =
0^\circ$ at $\alpha = 0^\circ$.
\item Stripe 86: $\alpha = 310.5^\circ$ to $\alpha = 60^\circ$, $\delta =
-10^\circ$ at $\alpha = 0^\circ$.
\end{enumerate}
See 
\begin{rawhtml}
<a href= http://www.astro.princeton.edu:81/sdss-southern/msg.76.html>here</a>\end{rawhtml}
for more precise definitions.  The equatorial stripe was chosen to
allow drift-scanning without moving the telescope, for the greatest
astrometric accuracy.  It will be observed multiple times, as
mentioned above.  The ``outrigger'' stripes were chosen to
maximize the number of long baselines for very large scale structure
studies. 

\item  Nominal PSF FWHM of the imaging survey: 1 arcsec\hfil\break
Basis: PSF should be limited by typical APO site seeing (0.8 arcsec),
combined with intrinsic PSF of the telescope optics.  See
\S~\ref{sec:depth} for more. 

\item The imaging limiting magnitude: $u'=22.3, g'= 23.3, r'=23.1, i'=22.3, z'=20.8$
(5$\sigma$ limit for point sources).\hfil\break
Basis:
\begin{enumerate}
\item This is the expected limiting magnitude for the camera and
telescope built with good engineering practice.
\item The selection of quasars for spectroscopy will involve $u'$ and
$z'$ 
photometry close to their respective photometric limits. 
\end{enumerate}

\item Roughly 150,000 QSO candidates will be targeted spectroscopically \\
Basis: The PoO defines the limiting magnitude for QSO spectroscopy to be
$r' = 19$, which yields an estimated 100,000 QSOs.  With an estimated
65\% success rate for QSO targeting, this yields the value above.  For
more detail on this and the following item, see \S~\ref{sec:target}.

\item The limiting magnitude for galaxy spectroscopy is $r' \approx
18$ (where the current plan is to use Petrosian magnitudes).

\item The imaging is done in a series of scans at the sidereal rate
drawing great circles 
in the sky.  The spacing between successive stripes should be
2.5$^\circ$ in $\eta$.
\end{enumerate}

\subsection{Basic parameters of the telescope}
See 
\begin{rawhtml}
<a href= http://www.apo.nmsu.edu/Telescopes/SDSS/SDSS.html>here</a>\end{rawhtml} for
details. 
\begin{enumerate}

\item Nominal optical design:
\begin{enumerate}
\item Primary mirror diameter: 2500 mm
\item Primary mirror focal ratio: 2.25
\item Secondary mirror diameter: 1140 mm
\item Nominal focal plane pixel scale (spectroscopic mode): 16.53 arcsec/mm
\item Primary mirror maximum $z$ position range: 13 mm; see 
(
\begin{rawhtml}
<a href= http://www.apo.nmsu.edu/Telescopes/SDSS/eng.papers/19971020_PlugPlateScale/19971020.html>here</a>\end{rawhtml})
\item Maximum range of scale adjustment: 0.7\% (same reference)
\item Focal plane radius (spectroscopic mode): 327 mm
\end{enumerate}

\item Mechanical parameters (weights, moment of inertia, etc.):\hfil\break
See 
\begin{rawhtml}
<a href= http://www.apo.nmsu.edu/Telescopes/SDSS/eng.papers/19970514_BasicTelescopeInfo/19970514.html>here</a>\end{rawhtml}

\item Telescope Axis Motions: \hfil\break
The following specs are taken from
\begin{rawhtml}
<a href=http://tdpc01.fnal.gov/sdss/docs/SDSSRequirements.htm>here</a>\end{rawhtml}, which is the
fundamental reference:
\begin{enumerate}
\item Pointing Precision: 3 arcsec, rms radius

\item Tracking Precision: The drop-dead requirement is 165 mas per
axis on time scales between 1 and 10 minutes.  It is believed that the
system can deliver 10 mas per axis in quiescent conditions, although
this is not a requirement.

\item Maximum tracking rate:
\begin{enumerate}
\item Azimuth: 45 arcsec/sec
\item Elevation: 15 arcsec/sec
\item Rotator: 45 arcsec/sec
\end{enumerate}

\item Absolute position transducer precision:
\begin{enumerate}
\item Altitude: 7.2 microns rms ($\pm 18$ microns at the transducer radius). 
\item Azimuth: 7.2 microns rms ($\pm 18$ microns at the transducer radius). 
\item Rotator: 30 microns rms  ($\pm 75$ microns at the transducer radius). 
\end{enumerate}

\item Normal operating range:
\begin{enumerate}
\item Altitude: 25-87$^\circ$
\item Azimuth: 90$^\circ\pm 270^\circ$
\item Rotator: 90$^\circ\pm 270^\circ$
\end{enumerate}

\item Normal velocity range: 0 to $2.8^\circ$/sec, all axes.

\item Maximum acceleration:
\begin{enumerate}
\item Azimuth: 50$^\circ/\rm sec^2$
\item Altitude: 50$^\circ/\rm sec^2$ 
\item Rotator: 100$^\circ/\rm sec^2$ 
\end{enumerate}

\item A specification on the motion of the secondary? 

\end{enumerate}
\end{enumerate}

\subsection{Basic parameters of the Photometric Telescope}

These parameters describe the recently installed 20'' telescope from
Johns Hopkins, and not the old 24'' Monitor Telescope, which has now
been decommissioned. 

\begin{enumerate}

\item Nominal optical design:
\begin{enumerate}
\item Primary mirror diameter: 500 mm
\item Primary mirror focal ratio: 3.00
\item Secondary mirror diameter: 200 mm
\item Nominal focal plane pixel scale (spectroscopic mode): 47.4 arcsec/mm
\item Focal plane radius 34.8 mm
\end{enumerate}

\end{enumerate}

\subsection{Nighttime Operating conditions}

\begin{enumerate}
\item Wind speed: $<30$ mph.\hfil\break
Basis: APO Operations shutdown limit
\item Temperature: $-20$ to +20 C.
\item $>1\mu$m dust counts below 3000, for at least 15 minutes.  See
the 
\begin{rawhtml}
<a href=http://www.apo.nmsu.edu/dustpolicy.html>APO dust
policy</a>\end{rawhtml}. 
\end{enumerate}


\subsection{Basic parameters of imaging camera}
\label{sec:camera-param}
The imaging camera was designed with the requirements of
\S~\ref{sec:depth} and \S~\ref{sec:uniform} in mind; see those
sections for further details.   See also the document, 
\begin{rawhtml}
<a href= http://www.astro.princeton.edu:81/requirements/camera_specifications.ps>here</a>\end{rawhtml}.


\begin{enumerate}

\item CCD active image area: $2048\times 2048$ pixels for photometric
chips, and $2048\times 400$ pixels for astrometric and focus chips.

\item Pixel size: 24 microns (0.4 arcsec).

\item Basic scan rate: 15 arcsec per sidereal second.

\item Number of CCDs:
\begin{enumerate}
\item Photometric: 30
\item Astrometric: 22 {\bf We could specify bridge and column chips separately.}
\item Focus: 2
\end{enumerate}

\item Transit time in drift scan mode:
\begin{enumerate}
\item Photometric CCDs: 55 sec
\item Astrometric and focus CCDs: 11 sec
\end{enumerate}

\item The CCD should have the quantum efficiencies given by Table 8.2 of
the Black Book. 

\item Stars in astrometric chips 5.5 mags brighter in $r'$ than
star in photometric chips should have the same signal-to-noise ratio,
in the brightness regime in which shot noise dominates the noise. 

\item Stars in focus chips 2.5 mags  brighter in $r'$ than
star in photometric chips should have the same signal-to-noise ratio. 

\item The read noise of the chips should match the limits listed in
Table 8.2 of the Black Book. 

\item The dark current for the photometric chips should be no more
than 6 electrons/pixel in 55 sec. For astrometric chips, no more than
60 electrons/pixel in 11 seconds.

\item The scatter in the pixel scale from chip to chip
should be no more than 20\% more than that in Table 6.2c (vscl). 

\item The charge transfer efficiency (CTE) should be as listed in page 8.27
of the Black Book. 

\item Center to center spacing of photometric CCD rows: 65 mm (17.98 arcmin)

\item Center to center spacing of photometric CCD columns: 91 mm (25.17 arcmin)

\item Overlap of scanlines within a stripe: 152 pixels (61 arcsec)

\item Overlap of scanlines between stripes: 402 pixels (161 arcsec)

\item Both astrometric and photometric chips should be rotationally
aligned to {0.25} pixel ({5}$\mu$m). Tilt error ${}< {25}\mu$m, total
piston errors ${}< {25}\mu$m.  

\item Temperature should be controlled to {1}$^\circ$C;
temperature on astrometric bench should be uniform to
{1}$^\circ$C.

\item Focus should be controlled to {20}$\mu$m rms at the secondary.

\item Filter effective wavelengths and widths:
\begin{enumerate}
\item $u'$: 3540 \AA, 570 \AA
\item $g'$: 4770 \AA, 1370 \AA
\item $r'$: 6230 \AA, 1370 \AA
\item $i'$: 7630 \AA, 1530 \AA
\item $z'$: 9130 \AA, 950 \AA
\end{enumerate}

\item {\bf Requirements on bias level, gain variations from chip to
chip, and the full-well level of the chips?}

\end{enumerate}

\subsection{Basic parameters of spectrographs}
\label{sec:spectro-params}
The spectrographs were designed with the requirements of
\S~\ref{sec:spectro} in mind; see there for justifications and
further details.  See also the document 
\begin{rawhtml}
<a href=
http://www.astro.princeton.edu:81/requirements/spectrograph_specifications.ps>here</a>\end{rawhtml}. 

\begin{enumerate}
\item Resolution: 1800-2000
\item The wavelength coverage should be continuous over the range $3900-9100$ \AA
\item Number of fibers: 640 (320 per spectrograph)
\item Fiber diameter: 180 microns
\item Wavelength coverage per channel:
\begin{enumerate}
\item Blue: $3900-6100$ \AA
\item Red: $5900-9100$ \AA
\end{enumerate}
\item Pixel size: 
\begin{enumerate}
\item Blue: 1.1 \AA
\item Red: 1.6 \AA
\end{enumerate}

\item Resolution: 3 pixels

\item Spectrograph demagnification: 2.5

\item Fiber spacing at slit-head: 390 microns (156 microns or 6.5 pixels at
CCD)

\item Dichroic crossover range: 200 \AA

\item Plug-plate thickness: 3.2 mm

\item Plug-plate diameter: 813 mm

\item Focal plane radius: 327 mm

\item CCD read noise: 5 electrons rms per pixel

\item CCD full well: 300,000 electrons per pixel {\it This may not be achievable}

\item Ferrule diameter: 2.154 mm (plugging bit); 3.17 mm (total)

\item Minimum fiber separation: 55 arcsec due to the physical diameter
of ferrule. 

\item Light trap holes: 3.175 mm diameter

\item Size of central post: 47.625 mm diameter
\end{enumerate}

\subsection{Basic Parameters of Spectrograph Guide Camera}
{\it There is some controversy on the numbers below; we should find
out the final numbers from Pat Waddell.}
\begin{enumerate}
\item Fiber diameter: 8 arcsec (9 bundles); 16 arcsec (1 bundle)

\item Scale: 60.6 arcsec/mm

\item Size: $384\times 256$ pixels (binned mode)

\item Pixel size: 18 microns
\end{enumerate}

\subsection{Data Acquisition System}

\begin{enumerate}
\item Monitor Telescope (sdssmth)

\begin{enumerate}
\item Cpu: SGI Crimson R4400, 150 Mhz

\item Memory: 144 Mbytes

\item Disk: $\approx$13 Gbytes

\item Tape drives: 2 DLT 2000
\end{enumerate}

\item 2.5 M telescope (sdsshost)

\begin{enumerate}
\item Cpu: SGI Crimson R4400, 150 Mhz

\item Memory: 144 Mbytes

\item Disk: $\approx$30 Gbytes

\item Tape drives: 2 DLT 2000
\end{enumerate}

\item VME Crates: Spectroscopy and Photometric Telescope
\begin{enumerate}

\item Nodes (CPU boards): 2

\item Frame Pool (Spectro) $\approx$50 frames

\item Frame Pool (PT) $\approx$?? frames

\item VCI+ boards: 3 (2 for spectro)
\end{enumerate}


\item VME Crates: Imager
\begin{enumerate}

\item Number of crates: 3

\item Nodes (CPU boards): 10 (2$\times$ 3 for photometrics; 4 for astrometrics)

\item Frame pool: 9 Gbytes per node

\item Capacity: 2.5 hours

\item Tape drives: 12 DLT 2000

\item Data rate to tape: 4.5 Mbytes/sec (total for photometrics).

\item Tape Capacity: Variable; 10-15 Gbytes (5 hours)

\item Cpu: 68040

\item Memory: 32 Mbytes per node

\item Frame Pool $\approx$ ?? frames
\end{enumerate}

\item Analysis Machine (sdss-commish)
\begin{enumerate} 
\item CPU: Dual PPro 200 Mhz

\item Memory: 256 Mbyte

\item Disk: 44 Gbyte
\end{enumerate}
\end{enumerate}

\subsection{What's Missing}

This section does not describe specifications on all aspects of
hardware.  In particular: 
\begin{enumerate} 
\item We do not have complete information on fibers and plugplates.
\item We have only limited specs on the imaging camera CCDs, both photometric
and astrometric. 
\item We have nothing on the PT camera.
\item We have nothing on the production system at Fermilab (although
this is covered somewhat in \S~\ref{sec:operations}). 
\end{enumerate}




\end{document}

