\documentclass{article}
\title{Proposed layout for DR7.2 - the legacy release}
\author{Sebastian Jester}
\begin{document}
\maketitle
\setcounter{section}{-1}

\section{Rules for the new web page}
\begin{itemize}
\item There shall be section numbers to the necessary hierarchy level
\item The nav-bar shall show clearly where in the hierarchy the
  current page is
\item The ``up'', ``next'', ``previous'' menubar items available in
  some browsers shall be usable for navigating the site
\item New top-level section names shall make it easy to find
  information already on DR6 page
\item Perhaps we should call this the Sloan Digital Sky Survey\textbf{s}?
\end{itemize}

\section{Overview \& other information}
This section introduces all top-level concepts and terms to give a
broad overview, and so that we can refer to all important terms
before they get explained in detail (e.g. CAS, spectroscopic samples,
etc.).

Not sure if we still need ``where to start'' as we are trying to be
comprehensive here. We do, however, need to list all the technical and
data release papers, and at the beginning is a good place.

\section{Surveys and coverage}

SDSS-I and SDSS-II, which actually contains 3 surveys:
\begin{enumerate}
\item Legacy -- the completion of SDSS-I imaging and
  spectroscopy. Includes ``special-plate'' spectroscopy in the
  southern galactic cap.
\item SEGUE -- the galactic-structure extension of SDSS-I. Some SEGUE
  imaging and spectra were taken as part of SDSS-I.
\item Supernova survey -- repeat scans under any conditions
\end{enumerate}

Now follows one subsection per survey, which explains:
\begin{itemize}
\item Motivation, scope
\item Survey contents, strategy, quality criteria (including top-level
  description of data products, i.e. imaging, spectra,
  calibrations...)
\item Sky and time coverage
\end{itemize}

\section{Instruments}

Pretty much as the existing one.

\section{Pipelines, algorithms \& products}

This again has one subsection per survey, since they are by now
sufficiently different from each other. Each explains which pipelines
there are, and what they do.
\begin{itemize}
\item Data flow overview
\item Photometry
  \begin{itemize}
  \item photo -- find objects and their properties in counts and
    pixels
  \item calibration pipelines and methods
    \begin{itemize}
    \item PSF
    \item photometric, flat-fielding, ubercal
    \item astrometric
    \end{itemize}
  \item Target selection, resolve concept of legacy, ``Primary''
    \begin{itemize}
    \item All the spectro surveys
    \end{itemize}
  \item tiling, masks
  \end{itemize}
\item Data model for each pipeline (though this is starting to get
  into data-access land); TARGET and BEST; we need to document all the
  flaws that appear somewhere in TARGET photometry!
\end{itemize}

\section{Data access}

As now

\section{Known unfixed problems}

As now, except that it's only a clearinghouse list with pointers to
detailed information further up -- doesn't need to be numbered,
though?

\section{Tutorials}

As now -- except with heavy updating!

\section{Help}

As now -- doesn't need to be numbered, though?

\section{Search}

As now -- doesn't need to be numbered, though?

\end{document}
